\documentclass[11pt, aspectratio=169]{beamer}

% --- PAQUETES DE COLOR Y TEMA ---
\usepackage{xcolor}
\usetheme{Madrid}

% Definimos el color antes de aplicarlo
\definecolor{AgroVerde}{RGB}{34, 100, 34}
\usecolortheme[named=AgroVerde]{structure}

% --- PAQUETES DE IDIOMA ---
\usepackage[utf8]{inputenc} % Para compatibilidad con pdflatex
\usepackage[T1]{fontenc}    % Codificación de fuente estándar
\usepackage[spanish, es-noquoting]{babel}
\usepackage{amsmath, amssymb}
\usepackage{graphicx}

% --- COMENTAMOS LAS FUENTES PERSONALIZADAS PARA EVITAR ERRORES ---
% Si quieres usarlas, asegúrate de que estén instaladas en tu sistema
% y de usar XeLaTeX. Por ahora, usamos las de defecto de Beamer.
% \usepackage{fontspec}
% \setmainfont{TeX Gyre Pagella}
% \setsansfont{DejaVu Sans}

% --- GRÁFICOS Y CÓDIGO ---
\usepackage{tikz}
\usepackage{pgfplots}
\usepackage{listings}
\pgfplotsset{compat=1.17}
\usetikzlibrary{arrows.meta, positioning, calc, babel}

% Estilo de código Python
\lstset{
    language=Python,
    basicstyle=\ttfamily\tiny,
    keywordstyle=\color{blue}\bfseries,
    commentstyle=\color{green!50!black},
    stringstyle=\color{orange},
    backgroundcolor=\color{gray!5},
    frame=single,
    rulecolor=\color{gray!30},
    breaklines=true,
    showstringspaces=false
}

% --- METADATOS ---
\title{\textbf{Modelado Predictivo en Agroindustria}}
\subtitle{De la Química del Suelo a la Predicción de Cosechas}
\author{Curso de IA Aplicada al Agro}
\date{Semana 04}
\institute{AgroFuture AI Training}

\begin{document}

\frame{\titlepage}

% --- DIAPOSITIVA 1 ---
\begin{frame}{¿Por qué predecir?}
    \begin{columns}
        \column{0.5\textwidth}
        \begin{block}{El dato es el nuevo fertilizante}
            \begin{itemize}
                \item \textbf{Descriptivo}: ¿Cuánto cosechamos? \\ \textit{\small (Análisis histórico)}
                \item \textbf{Predictivo}: ¿Cuánto cosecharíamos si...? \\ \textit{\small (Modelos de IA)}
            \end{itemize}
        \end{block}

        \column{0.5\textwidth}
        \begin{exampleblock}{Caso: Finca ``La Esperanza''}
            Gasto en fertilizantes $\uparrow$ 20\%, pero producción estancada.\\
            \textbf{Meta}: Encontrar la dosis óptima de nitrógeno (N).
        \end{exampleblock}
    \end{columns}
\end{frame}

% --- DIAPOSITIVA 2 ---
\begin{frame}{Regresión Lineal: La Hipótesis}
    \begin{center}
        \Large $h_\theta(x) = \theta_0 + \theta_1 x$
    \end{center}
    \vspace{0.3cm}
    \begin{columns}[T]
        \column{0.5\textwidth}
        \textbf{$\theta_0$ (Intercepto)} \\
        Rendimiento base esperado si el nivel de nitrógeno es cero.

        \column{0.5\textwidth}
        \textbf{$\theta_1$ (Pendiente)} \\
        El ``peso'' del nitrógeno: Toneladas extra por cada unidad de ppm.
    \end{columns}
    \vspace{0.4cm}
    \begin{alertblock}{Misión del Algoritmo}
        Ajustar los parámetros $\theta$ para que la línea minimice la distancia a los puntos reales registrados en el campo.
    \end{alertblock}
\end{frame}

% --- DIAPOSITIVA 3 ---
\begin{frame}{Visualización del Error}
    \begin{center}
        \begin{tikzpicture}
            \begin{axis}[
                axis lines=left, xlabel={Nitrógeno (ppm)}, ylabel={Rendimiento (Ton/Ha)},
                ymin=0, ymax=5, xmin=0, xmax=10, grid=major, width=9cm, height=5.5cm,
                grid style={dashed, gray!30}
            ]
                \addplot[only marks, mark=*, color=black] coordinates {
                    (1, 1.2) (2, 1.8) (3, 2.5) (4, 3.0) (5, 3.2) (6, 4.1) (8, 4.5)
                };
                \addplot[domain=0:9, color=AgroVerde, ultra thick] {0.8 + 0.45*x};
                \draw[red, thick, dashed] (axis cs:4,3.0) -- (axis cs:4,2.6);
                \node[red, right] at (axis cs:4,2.8) {\footnotesize Residuo (Error)};
            \end{axis}
        \end{tikzpicture}
    \end{center}
\end{frame}

% --- DIAPOSITIVA 4 ---
\begin{frame}{Descenso del Gradiente}
    \textbf{¿Cómo aprende el modelo?}
    \vspace{0.3cm}
    \begin{columns}
        \column{0.6\textwidth}
        \begin{enumerate}
            \item Calcula el error actual.
            \item Determina la dirección de descenso (\textbf{Gradiente}).
            \item Actualiza los pesos con un paso ($\alpha$).
        \end{enumerate}
        \begin{block}{Regla de Oro}
            \[ \theta_j := \theta_j - \alpha \frac{\partial J}{\partial \theta_j} \]
        \end{block}
        \column{0.4\textwidth}
        \centering
        \begin{tikzpicture}[scale=0.5]
            \draw[->] (0,4) .. controls (1,0) and (3,0) .. (4,4);
            \filldraw[red] (0.5,2.5) circle (4pt);
            \draw[->, thick] (0.5,2.5) -- (1.2,1.2);
            \node at (2,-0.5) {Mínimo Error};
        \end{tikzpicture}
    \end{columns}
\end{frame}



% --- DIAPOSITIVA 5 ---
\begin{frame}[fragile]{Implementación Profesional}
\begin{lstlisting}
from sklearn.linear_model import LinearRegression

# Entrenar modelo con datos de campo
modelo = LinearRegression()
modelo.fit(X_entrenamiento, y_real)

# Obtener la precision del modelo
r2 = modelo.score(X_prueba, y_real_prueba)
print(f"Confiabilidad del modelo: {r2:.2%}")

# Predecir cosecha para un nuevo lote
pred = modelo.predict([[3.5]])
\end{lstlisting}
    \begin{exampleblock}{Interpretación}
        Si el $R^2$ es 0.95, significa que el nitrógeno explica el 95\% de la variación en tu cosecha. Es un modelo excelente para decidir la compra de insumos.
    \end{exampleblock}
\end{frame}

% --- DIAPOSITIVA 6 ---
\begin{frame}{Conclusiones y Futuro}
    \begin{itemize}
        \item \textbf{Criterio Humano}: Ningún modelo sustituye al agrónomo. Cuidado con la extrapolación.
        \item \textbf{Ética}: Modelos precisos evitan el uso excesivo de fertilizantes (menor impacto ambiental).
        \item \textbf{Próxima Semana}: ¿Qué pasa si sumamos la lluvia y el pH del suelo? (\textbf{Regresión Múltiple}).
    \end{itemize}
    \vfill
    \centering
    \Large \textbf{¡Gracias por su atención!} \\
    \normalsize Espacio para preguntas y respuestas.
\end{frame}

\end{document}