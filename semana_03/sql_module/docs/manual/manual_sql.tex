\documentclass[a4paper,11pt]{article}
\usepackage{fontspec}
\usepackage[spanish, es-tabla]{babel} % es-tabla para que diga "Tabla" en vez de "Cuadro"
\usepackage[margin=2.5cm]{geometry}
\usepackage{xcolor}
\usepackage{listings}
\usepackage[most]{tcolorbox}
\usepackage{hyperref}
\usepackage{graphicx}
\usepackage{fancyhdr}
\usepackage{booktabs}
\usepackage{float} % Para forzar posición de imágenes con [H]

% --- Fuentes (Requiere compilar con XeLaTeX) ---
% Si no tienes las fuentes instaladas, puedes comentar estas líneas y usar las default
\setmainfont{DejaVu Sans}
\setmonofont{DejaVu Sans Mono}

% --- Colores Corporativos Agro ---
\definecolor{sqlblue}{RGB}{0, 102, 204}
\definecolor{agrogreen}{RGB}{34, 139, 34}
\definecolor{earthbrown}{RGB}{139, 69, 19}
\definecolor{danger}{RGB}{204, 0, 0}
\definecolor{codebg}{RGB}{245, 247, 250}

% --- Configuración de Hipervínculos ---
\hypersetup{
    colorlinks=true,
    linkcolor=sqlblue,
    filecolor=magenta,
    urlcolor=agrogreen,
}

% --- Estilo de Código SQL y Bash ---
\lstdefinestyle{estilo_codigo}{
    backgroundcolor=\color{codebg},
    basicstyle=\ttfamily\small,
    keywordstyle=\color{sqlblue}\bfseries,
    commentstyle=\color{gray}\itshape,
    stringstyle=\color{agrogreen},
    numberstyle=\tiny\color{gray},
    numbers=left,
    stepnumber=1,
    frame=single,
    rulecolor=\color{lightgray},
    breaklines=true,
    showstringspaces=false,
    captionpos=b
}
\lstset{style=estilo_codigo}

% --- Cajas Personalizadas ---
\newtcolorbox{agrobox}[1]{
  colback=green!5!white, colframe=agrogreen,
  title=🌱 #1, fonttitle=\bfseries,
  boxrule=0.5mm, arc=2mm, drop shadow
}

\newtcolorbox{conceptbox}[1]{
  colback=blue!5!white, colframe=sqlblue,
  title=🧠 Concepto Clave: #1, fonttitle=\bfseries,
  boxrule=0.5mm, arc=2mm, drop shadow
}

\newtcolorbox{warningbox}[1]{
  colback=red!5!white, colframe=danger,
  title=⚠️ Advertencia: #1, fonttitle=\bfseries,
  boxrule=0.5mm, arc=2mm
}

% --- Encabezado y Pie de Página ---
\pagestyle{fancy}
\fancyhf{}
\lhead{\textbf{Ingeniería de Datos Agro}}
\rhead{Semana 04: Persistencia de Datos}
\rfoot{Página \thepage}
\renewcommand{\headrulewidth}{1pt}
\renewcommand{\footrulewidth}{0.5pt}

% --- Título ---
\title{
    \vspace{-2cm}
    \begin{tcolorbox}[colback=agrogreen!10!white,colframe=agrogreen,arc=0mm,boxrule=0pt]
        \centering
        \huge \textbf{SQL Moderno para Ingeniería Agroindustrial}\\
        \Large Integración con Bash, Python y Pandas
    \end{tcolorbox}
}
\author{Curso de IA Aplicada al Agro}
\date{\today}

\begin{document}

\maketitle
\tableofcontents
\newpage

\section*{Introducción: El Flujo de Datos}

En la Semana 03 aprendiste a limpiar datos en memoria con Pandas. Pero ¿y si los datos son demasiado grandes? ¿O si necesitas acceder a ellos desde múltiples scripts simultáneamente?

La respuesta es: ``almacénalos en una base de datos''.

\begin{figure}[H]
    \centering
    % Aquí podrías incluir un diagrama real si tienes el archivo .png
    % \includegraphics[width=0.8\textwidth]{pipeline.png}
    \begin{tcolorbox}[colback=white, colframe=gray, title=Arquitectura del Pipeline, halign=center]
        \textbf{Sensores (CSV)} $\xrightarrow{\text{Bash}}$ \textbf{SQLite (DB)} $\xrightarrow{\text{Python}}$ \textbf{Pandas (DF)}
    \end{tcolorbox}
    \caption{Flujo de datos desde la recolección hasta el análisis.}
\end{figure}

SQL (Structured Query Language) es el estándar porque:
\begin{itemize}
    \item Es \textbf{declarativo}: describes \textit{qué} quieres, no \textit{cómo} hacerlo.
    \item Es \textbf{eficiente}: filtra millones de registros antes de cargarlos en la RAM de Python.
    \item Es \textbf{universal}: lo que aprendas aquí sirve para PostgreSQL, BigQuery o Spark SQL.
\end{itemize}

\newpage

\section{Capítulo I: Primeros Pasos con SQLite}

SQLite es una base de datos ligera, sin servidor, ideal para entornos frugales (Raspberry Pi, laptops rurales) donde la conectividad es limitada.

\subsection{Crear y Poblar una Tabla}
Imagina una tabla que registra la actividad de sensores de humedad en diferentes lotes de café.



\begin{lstlisting}[language=SQL, caption=DDL para crear la tabla]
-- Crear tabla de sensores
CREATE TABLE sensores (
    id INTEGER PRIMARY KEY,
    zona TEXT NOT NULL,
    fecha DATE,
    temperatura REAL,
    humedad REAL
);

-- Insertar datos manuales (solo para prueba)
INSERT INTO sensores (zona, fecha, temperatura, humedad)
VALUES ('Norte', '2026-01-10', 24.5, 78.2);
\end{lstlisting}

\subsection{Consultas Básicas (Queries)}
\begin{lstlisting}[language=SQL, caption=Consultas de exploración]
-- 1. Ver todos los registros
SELECT * FROM sensores;

-- 2. Filtrar zonas críticas (Humedad alta = riesgo de hongos)
SELECT zona, fecha
FROM sensores
WHERE humedad > 80;

-- 3. Ordenar por temperatura
SELECT * FROM sensores ORDER BY temperatura DESC;
\end{lstlisting}

\begin{agrobox}{Caso de Uso: Alerta de Roya}
La consulta número 2 es vital. Si la humedad supera el 80\% y la temperatura está entre 20-25°C, las condiciones son ideales para la roya del café. SQL nos permite detectar esto en milisegundos.
\end{agrobox}

\newpage

\section{Capítulo II: Ingesta Masiva desde Bash}

Como ingenieros de datos, no insertamos datos uno por uno. Automatizamos la ingesta de archivos CSV generados por los dataloggers.

\subsection{Importar CSV a SQLite}
Supón que tienes \texttt{sensores\_2026.csv} con miles de registros.

\begin{lstlisting}[language=bash, caption=Script de Bash para Ingesta]
# Comando en terminal (Bash)
# .mode csv le dice a SQLite que espere comas
# .import archivo tabla
sqlite3 finca.db << EOF
.mode csv
.import sensores_2026.csv sensores
EOF
\end{lstlisting}

\begin{warningbox}{Integridad de Datos}
Asegúrate de que la primera fila del CSV (encabezados) coincida con los nombres de las columnas en la tabla, o usa la opción \texttt{--skip 1} si vas a mapear columnas manualmente.
\end{warningbox}

\subsection{Exportar Resultados a CSV}
Después de procesar los datos, el agrónomo necesita un Excel.

\begin{lstlisting}[language=bash]
sqlite3 finca.db << EOF
.headers on
.mode csv
.output reporte_sequia.csv
SELECT zona, AVG(humedad) as prom_hum
FROM sensores
GROUP BY zona
HAVING prom_hum < 40;
EOF
\end{lstlisting}

\newpage

\section{Capítulo III: SQL + Python = Poder Total}

Pandas delega el trabajo pesado a SQL y se encarga del análisis fino y la visualización.

\subsection{Conectar y Consultar}
\begin{lstlisting}[language=Python, caption=Python leyendo SQL]
import sqlite3
import pandas as pd

# 1. Conexión a la base de datos (archivo local)
conn = sqlite3.connect('finca.db')

# 2. Ejecutar consulta optimizada
# Traemos SOLO los datos necesarios, no toda la base
query = '''
    SELECT zona, AVG(temperatura) as temp_prom
    FROM sensores
    WHERE fecha >= '2026-01-01'
    GROUP BY zona
'''

df = pd.read_sql(query, conn)

print("--- Promedios por Zona ---")
print(df)

conn.close()
\end{lstlisting}

\begin{conceptbox}{Eficiencia de Memoria}
Filtrar en SQL (\texttt{WHERE}) es 10x más eficiente que cargar todo el CSV en Pandas y filtrar después (\texttt{df[df['col'] > x]}).
\end{conceptbox}

\newpage

\section{Capítulo IV: Ética y Gobernanza}

Una base de datos no es neutral. Refleja decisiones de diseño que impactan a las personas.

\begin{itemize}
    \item \textbf{¿Qué se mide?} Si solo medimos producción y no salud del suelo, optimizaremos la explotación a corto plazo.
    \item \textbf{¿Quién accede?} El pequeño agricultor debe ser dueño de sus métricas, no solo la empresa que vende los insumos.
\end{itemize}

\begin{agrobox}{Soberanía Tecnológica}
En contextos rurales, evita depender 100\% de la nube. SQLite permite mantener los datos \textbf{locales, controlables y auditables} incluso si se corta el internet satelital.
\end{agrobox}

\section*{🏆 Reto Final: Monitor de Cultivos}

\begin{enumerate}
\item Ejecuta \texttt{generar\_sensores.py} para crear \texttt{datos\_crudos.csv}.
\item Crea la base de datos \texttt{finca.db} y la tabla \texttt{lecturas} usando Bash.
\item Escribe una consulta SQL que identifique "Días Críticos":
    * Humedad < 30\% (Estrés hídrico)
    * **O** Temperatura > 35°C (Golpe de calor)
\item Carga esos datos en Python y genera un gráfico de barras.
\item Exporta los IDs de los lotes afectados a \texttt{alertas.txt}.
\end{enumerate}
\end{document}