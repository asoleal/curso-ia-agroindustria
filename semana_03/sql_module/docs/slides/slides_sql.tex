\documentclass[aspectratio=169, 10pt]{beamer}

% --- Paquetes ---
\usepackage[spanish]{babel}
\usepackage{fontspec}
\usepackage{listings}
\usepackage{xcolor}
\usepackage{tikz}
\usetikzlibrary{arrows.meta, positioning}

% --- Tema y Colores ---
\usetheme{Madrid}
\usecolortheme{default}

% Definición de colores del Manual
\definecolor{sqlblue}{RGB}{0, 102, 204}
\definecolor{agrogreen}{RGB}{34, 139, 34}
\definecolor{earthbrown}{RGB}{139, 69, 19}
\definecolor{codebg}{RGB}{245, 247, 250}

% Personalización de Beamer con colores Agro
\setbeamercolor{structure}{fg=agrogreen}
\setbeamercolor{frametitle}{bg=agrogreen, fg=white}
\setbeamercolor{title}{bg=agrogreen, fg=white}
\setbeamercolor{block title}{bg=sqlblue, fg=white}
\setbeamercolor{alerted text}{fg=earthbrown}

% --- Fuentes (Para compilar con XeLaTeX) ---
\setsansfont{DejaVu Sans}
\setmonofont{DejaVu Sans Mono}

% --- Configuración de Código ---
\lstdefinestyle{slidecode}{
    backgroundcolor=\color{codebg},
    basicstyle=\ttfamily\scriptsize,
    keywordstyle=\color{sqlblue}\bfseries,
    commentstyle=\color{gray}\itshape,
    stringstyle=\color{agrogreen},
    numberstyle=\tiny\color{gray},
    numbers=left,
    frame=single,
    breaklines=true,
    showstringspaces=false
}
\lstset{style=slidecode}

% --- Metadatos ---
\title{SQL Moderno para Ingeniería Agroindustrial}
\subtitle{Semana 04: Integración Bash + SQL + Python}
\author{Ingeniería de Datos}
\date{\today}

\begin{document}

% -------------------------------------------------------------------
% SLIDE 1: PORTADA
% -------------------------------------------------------------------
\begin{frame}
    \titlepage
    \begin{center}
        \footnotesize \textit{``Los datos son el fertilizante de la agricultura moderna.''}
    \end{center}
\end{frame}

% -------------------------------------------------------------------
% SLIDE 2: INTRODUCCIÓN
% -------------------------------------------------------------------
\begin{frame}{¿Por qué SQL en 2026?}
    \begin{columns}
        \column{0.5\textwidth}
        \textbf{El problema con los CSV/Excel:}
        \begin{itemize}
            \item Se cargan completos en RAM.
            \item Difícil cruzar información (VLOOKUP es lento).
            \item Sin control de tipos de datos.
        \end{itemize}

        \column{0.5\textwidth}
        \textbf{La solución SQL:}
        \begin{itemize}
            \item \textbf{Eficiente:} Procesa en disco, no en RAM.
            \item \textbf{Relacional:} Une tablas complejas fácilmente.
            \item \textbf{Estándar:} Funciona igual en una Raspberry Pi que en la Nube.
        \end{itemize}
    \end{columns}

    \vspace{1cm}
    \begin{alertblock}{En el Agro}
        Los sensores generan datos 24/7. SQL es el silo donde guardamos esa cosecha digital.
    \end{alertblock}
\end{frame}

% -------------------------------------------------------------------
% SLIDE 3: ARQUITECTURA (VISUAL)
% -------------------------------------------------------------------
\begin{frame}{El Pipeline de Datos}
    \centering

    \begin{tikzpicture}[node distance=2cm, auto,
        block/.style={rectangle, draw, fill=white, text width=2.5cm, align=center, rounded corners, minimum height=1cm},
        arrow/.style={-Stealth, thick, color=agrogreen}
    ]
        % Nodos
        \node[block, fill=gray!10] (raw) {\textbf{Sensores}\\(Archivos CSV)};
        \node[block, fill=sqlblue!10, right=of raw] (db) {\textbf{SQLite}\\(Base de Datos)};
        \node[block, fill=agrogreen!10, right=of db] (py) {\textbf{Python}\\(Análisis/IA)};

        % Flechas
        \draw[arrow] (raw) -- node[above, font=\tiny] {\texttt{BASH (.import)}} (db);
        \draw[arrow] (db) -- node[above, font=\tiny] {\texttt{SQL (SELECT)}} (py);

    \end{tikzpicture}

    \vspace{1cm}
    \begin{itemize}
        \item \textbf{Bash:} El cargador rápido (Ingesta).
        \item \textbf{SQL:} El almacén ordenado (Persistencia).
        \item \textbf{Python:} El cerebro analítico (Insights).
    \end{itemize}
\end{frame}

% -------------------------------------------------------------------
% SLIDE 4: CAPÍTULO I - SQL BÁSICO
% -------------------------------------------------------------------
\begin{frame}[fragile]{Capítulo I: Estructurando el Campo}
    Antes de sembrar datos, debemos preparar la tierra (Crear Tablas).



    \begin{lstlisting}[language=SQL, title={DDL - Data Definition Language}]
-- Crear la bitácora de sensores
CREATE TABLE sensores (
    id INTEGER PRIMARY KEY,
    zona TEXT NOT NULL,      -- Ej: 'Lote Norte'
    fecha DATETIME,
    temperatura REAL,        -- Grados Celsius
    humedad REAL             -- Porcentaje %
);
    \end{lstlisting}

    \pause
    \vspace{0.5cm}
    \textbf{Consultando datos críticos:}
    \begin{lstlisting}[language=SQL]
SELECT zona, fecha
FROM sensores
WHERE humedad > 80 AND temperatura > 25;
-- Esto detecta condiciones ideales para hongos
    \end{lstlisting}
\end{frame}

% -------------------------------------------------------------------
% SLIDE 5: CAPÍTULO II - BASH Y CARGA MASIVA
% -------------------------------------------------------------------
\begin{frame}[fragile]{Capítulo II: Ingesta Masiva con Bash}
    \textbf{Situación:} Tienes 50 archivos CSV de sensores. No vas a hacer \texttt{INSERT} manual.

    Usamos \texttt{sqlite3} en la terminal:

    \begin{block}{Comando de Terminal (ETL Express)}
    \begin{lstlisting}[language=bash]
# 1. Definir modo CSV
# 2. Importar archivo -> tabla
sqlite3 finca.db << EOF
.mode csv
.import datos_crudos.csv sensores
EOF
    \end{lstlisting}
    \end{block}

    \vspace{0.2cm}
    \small \textit{*Esto es 100 veces más rápido que un bucle for en Python.}
\end{frame}

% -------------------------------------------------------------------
% SLIDE 6: CAPÍTULO III - PYTHON
% -------------------------------------------------------------------
\begin{frame}[fragile]{Capítulo III: Conexión con Pandas}
    Una vez los datos están en SQL, Python solo trae lo necesario.



    \begin{lstlisting}[language=Python]
import sqlite3
import pandas as pd

conn = sqlite3.connect('finca.db')

# BUENA PRÁCTICA: Filtrar en SQL, no en Pandas
query = """
    SELECT zona, AVG(temperatura) as media
    FROM sensores
    WHERE fecha >= '2024-01-01'
    GROUP BY zona
"""

df = pd.read_sql(query, conn)
print(df)
    \end{lstlisting}
\end{frame}

% -------------------------------------------------------------------
% SLIDE 7: ÉTICA Y SOBERANÍA
% -------------------------------------------------------------------
\begin{frame}{Ética: Soberanía de Datos}
    \begin{alertblock}{¿Dónde viven los datos?}
        Si usamos Google Sheets o AWS, dependemos de internet y corporaciones extranjeras.
    \end{alertblock}

    \vspace{0.5cm}
    \textbf{Ventajas de SQLite en el Agro:}
    \begin{enumerate}
        \item \textbf{Local:} Funciona `` Offline'' en la finca.
        \item \textbf{Propiedad:} El archivo `.db` es fácil de copiar y respaldar.
        \item \textbf{Auditabilidad:} El agricultor puede ver qué se registra.
    \end{enumerate}
\end{frame}

% -------------------------------------------------------------------
% SLIDE 8: RETO FINAL
% -------------------------------------------------------------------
\begin{frame}{🏆 Taller: Monitor de Cultivos}
    \textbf{Objetivo:} Construir el Pipeline completo.

    \begin{enumerate}
        \item \textbf{Generar Caos:} Ejecuta \texttt{generar\_sensores.py} para crear datos sucios.
        \item \textbf{Ingesta (Bash):} Crea la DB e importa el CSV automáticamente.
        \item \textbf{Lógica (SQL):} Encuentra los "Días de Estrés Hídrico" (Humedad < 30\%).
        \item \textbf{Reporte (Python):} Grafica los lotes más afectados.
    \end{enumerate}

    \vspace{1cm}
    \centering
    \textbf{¡Manos a la obra!} 🚀
\end{frame}

\end{document}