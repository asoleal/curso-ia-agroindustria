\documentclass[aspectratio=169, 10pt]{beamer}

% --- TEMA ---
% CORRECCIÓN: Cargamos el tema limpio, sin parámetros inventados.
% (Dará una advertencia en el log sobre Fira Sans, pero compilará bien
% porque sobrescribimos la fuente justo abajo).
\usetheme{metropolis}

% --- FUENTES (DejaVu Sans - Instalada en tu sistema) ---
\usepackage{fontspec}
\setsansfont{DejaVu Sans}[
    BoldFont={DejaVu Sans Bold},
    ItalicFont={DejaVu Sans Oblique},
    Scale=0.9
]
\setmonofont{DejaVu Sans Mono}[Scale=0.8]

% --- IDIOMA ---
\usepackage{polyglossia}
\setmainlanguage{spanish}

% --- PAQUETES ---
\usepackage{xcolor}
\usepackage{listings}
\usepackage{booktabs}
\usepackage{tikz}
\usepackage{tcolorbox}
\usetikzlibrary{positioning, shapes, arrows, shadows}

% --- COLORES DE TU MARCA ---
\definecolor{primary}{RGB}{0, 85, 164}      % Azul Ingeniería
\definecolor{accent}{RGB}{34, 139, 34}      % Verde Agro
\definecolor{danger}{RGB}{204, 0, 0}        % Rojo Alerta
\definecolor{pandas}{RGB}{19, 7, 84}        % Azul oscuro (Pandas)
\definecolor{codebg}{RGB}{240, 240, 240}

% Configuración de colores de Metropolis
\setbeamercolor{frametitle}{bg=primary, fg=white}
\setbeamercolor{progress bar}{fg=accent, bg=gray!20}
\setbeamercolor{alerted text}{fg=danger}
\setbeamercolor{block title}{bg=primary, fg=white}

% --- ESTILO DE CÓDIGO ---
\lstdefinestyle{slidecode}{
    language=Python,
    backgroundcolor=\color{codebg},
    commentstyle=\color{gray},
    keywordstyle=\color{pandas}\bfseries,
    numberstyle=\tiny\color{gray},
    stringstyle=\color{accent},
    basicstyle=\ttfamily\scriptsize,
    breaklines=true,
    frame=none,
    numbers=none,
    showstringspaces=false,
    literate={á}{{\'a}}1 {é}{{\'e}}1 {í}{{\'i}}1 {ó}{{\'o}}1 {ú}{{\'u}}1 {ñ}{{\~n}}1
}
\lstset{style=slidecode}

% --- INFORMACIÓN ---
\title{Domando Datos Reales con Pandas}
\subtitle{Limpieza, Análisis y Ética en el Agro}
\date{\today}
\author{Ingeniería de Software I}
\institute{Curso de IA Aplicada}

\begin{document}

% -------------------------------------------------------------------
% SLIDE 1: PORTADA
% -------------------------------------------------------------------
\maketitle

% -------------------------------------------------------------------
% SLIDE 2: MOTIVACIÓN
% -------------------------------------------------------------------
\begin{frame}{¿Por qué NumPy no es suficiente?}
    En la vida real, los datos son \textbf{sucios} y \textbf{heterogéneos}.

    \vspace{0.5cm}

    \begin{columns}
        \column{0.5\textwidth}
        \textbf{El problema de NumPy:}
        \begin{itemize}
            \item Solo maneja números.
            \item No sabe qué es una fecha.
            \item Si falta un dato, todo falla.
            \item Acceso por índice opaco: \texttt{data[0, 4]}
        \end{itemize}

        \column{0.5\textwidth}
        \textbf{La solución Pandas:}
        \begin{itemize}
            \item Etiquetas: \texttt{data['Humedad']}.
            \item Series Temporales nativas.
            \item Manejo robusto de \texttt{NaN}.
            \item \textbf{SQL para Python.}
        \end{itemize}
    \end{columns}
\end{frame}

% -------------------------------------------------------------------
% SLIDE 3: ESTRUCTURA VISUAL (TIKZ)
% -------------------------------------------------------------------
\begin{frame}{Anatomía: Series vs DataFrame}
    \centering
    \begin{tikzpicture}[node distance=1cm]
        % Series
        \node (series) [draw=accent, fill=green!10, rectangle, rounded corners, minimum width=2.5cm, minimum height=3cm, align=center, drop shadow] {
            \textbf{Series (1D)}\\
            \scriptsize Columna\\
            \rule{2cm}{0.5pt}\\
            \scriptsize 10.5\\
            \scriptsize 11.2\\
            \scriptsize NaN\\
            \scriptsize \textit{Homogéneo}
        };

        % Arrow
        \draw[->, ultra thick, gray] (series) -- ++(3,0) node[midway, above] {\footnotesize Se agrupan en};

        % DataFrame
        \node (df) [right=of series, xshift=2cm, draw=primary, fill=blue!10, rectangle, rounded corners, minimum width=4cm, minimum height=3cm, align=center, drop shadow] {
            \textbf{DataFrame (2D)}\\
            \scriptsize Tabla (Excel Potente)\\
            \rule{3.5cm}{0.5pt}\\
            \scriptsize Índice + Columnas Variadas\\
            \scriptsize (Float, String, Date)
        };
    \end{tikzpicture}
\end{frame}

% -------------------------------------------------------------------
% SLIDE 4: CARGA DE DATOS (CODIGO)
% -------------------------------------------------------------------
\begin{frame}[fragile]{Ingesta de Datos (I/O)}
    No confíes en los valores por defecto. Define tus tipos.

    \begin{lstlisting}
import pandas as pd

# Carga robusta para Agro
df = pd.read_csv(
    'cosecha_2024.csv',
    sep=';',                     # Delimitador comun en Latam
    parse_dates=['fecha'],       # Interpretar tiempo
    na_values=['-', 'error'],    # Convertir basura a NaN
    index_col='fecha'            # Usar fecha como eje X
)

# Inspeccion rapida
print(df.info())
    \end{lstlisting}

    \begin{alertblock}{Tip Pro}
        Si una columna numérica tiene un solo texto, Pandas la volverá texto (\texttt{object}). Usa \texttt{.info()} siempre.
    \end{alertblock}
\end{frame}

% -------------------------------------------------------------------
% SLIDE 5: SELECCIÓN
% -------------------------------------------------------------------
\begin{frame}[fragile]{Navegación: .loc vs .iloc}
    La confusión \#1 en entrevistas de trabajo.

    \vspace{0.5cm}
    \centering
    \begin{tabular}{llp{5cm}}
        \toprule
        \textbf{Comando} & \textbf{Busca por...} & \textbf{Ejemplo} \\
        \midrule
        \texttt{.loc[]} & \textbf{Etiqueta} (Nombre) & \texttt{df.loc['2024-01', 'pH']} \\
        \texttt{.iloc[]} & \textbf{Posición} (Índice) & \texttt{df.iloc[0, 4]} \\
        \bottomrule
    \end{tabular}

    \vspace{0.5cm}
    \begin{lstlisting}
# Filtrado Booleano (Mascara)
# "Dias calurosos con humedad alta"
alerta = df[ (df['temp'] > 30) & (df['hum'] > 80) ]
    \end{lstlisting}
\end{frame}

% -------------------------------------------------------------------
% SLIDE 6: LIMPIEZA
% -------------------------------------------------------------------
\begin{frame}[fragile]{Limpieza y Valores Nulos}
    En el campo, los sensores fallan. ¿Qué hacemos con los huecos?

    \begin{itemize}
        \item \textbf{Borrar:} \texttt{dropna()} $\rightarrow$ Perdemos datos valiosos.
        \item \textbf{Promedio:} \texttt{fillna(mean)} $\rightarrow$ Aplana la curva.
        \item \textbf{Interpolación:} La opción correcta para series temporales.
    \end{itemize}

    \begin{lstlisting}
# Relleno inteligente (conecta los puntos)
df['humedad'] = df['humedad'].interpolate(method='time')

# Sanear errores fisicos
# Humedad > 100% es imposible -> convertir a NaN
df.loc[df['humedad'] > 100, 'humedad'] = pd.NA
    \end{lstlisting}
\end{frame}

% -------------------------------------------------------------------
% SLIDE 7: AGREGACIÓN
% -------------------------------------------------------------------
\begin{frame}[fragile]{Análisis: GroupBy y Resample}
    \textbf{GroupBy:} Para categorías (Lotes, Variedades).
    \textbf{Resample:} Para tiempo (Días, Meses).

    \begin{lstlisting}
# ¿Cual lote produce mas?
rendimiento = df.groupby('id_lote')['kilos'].sum()

# Promedio mensual de clima
clima_mes = df.resample('M').agg({
    'temp': 'mean',
    'lluvia': 'sum'
})
    \end{lstlisting}
\end{frame}

% -------------------------------------------------------------------
% SLIDE 8: ÉTICA
% -------------------------------------------------------------------
\begin{frame}{Ética del Dato}
    Limpiar datos es \textbf{alterar la realidad}.

    \begin{block}{Preguntas Obligatorias}
        \begin{enumerate}
            \item \textbf{Trazabilidad:} ¿Guardo el original \texttt{raw}?
            \item \textbf{Sesgo:} Al rellenar con promedios, ¿oculto fallas sistémicas en zonas vulnerables?
            \item \textbf{Transparencia:} ¿Está documentado mi proceso de limpieza?
        \end{enumerate}
    \end{block}
\end{frame}

% -------------------------------------------------------------------
% SLIDE 9: EJERCICIO EN VIVO
% -------------------------------------------------------------------
\begin{frame}[fragile]{Ejercicio Rápido: Calidad de Leche}
    Detectar vacas con problemas de salud (pH fuera de 6.6 - 6.8).

    \begin{lstlisting}
data = {
    'vaca': ['V1', 'V2', 'V3'],
    'ph':   [6.7,  6.2,  7.1],  # V2 acida, V3 alcalina
    'lts':  [20,   18,   15]
}
df = pd.DataFrame(data)

# 1. Filtrar anomalías
enfermas = df[ (df['ph'] < 6.6) | (df['ph'] > 6.8) ]

# 2. Calcular perdida economica (precio = 2000)
perdida = enfermas['lts'].sum() * 2000
print(f"Perdida hoy: ${perdida}")
    \end{lstlisting}
\end{frame}

% -------------------------------------------------------------------
% SLIDE 10: CIERRE
% -------------------------------------------------------------------
\begin{frame}{Próximos Pasos}
    \centering
    \Large
    ¿Listos para el reto semanal?

    \vspace{1cm}
    \normalsize
    Descarguen \texttt{clima\_corrupto.csv} del repositorio.

    \vspace{0.5cm}
    \textbf{¡A programar!}
\end{frame}

\end{document}