\documentclass[12pt, a4paper]{article}
\usepackage[utf8]{inputenc}
\usepackage[spanish]{babel}
\usepackage{geometry}
\usepackage{xcolor}
\usepackage{listings}
\usepackage{tcolorbox}
\usepackage{longtable}
\usepackage{amssymb} % Para simbolos matematicos

\geometry{top=2.5cm, bottom=2.5cm, left=2.5cm, right=2.5cm}

\title{\textbf{Manual de Ingeniería I}\\ Fundamentos de Terminal y Automatización}
\author{Semana 01 - IA Agroindustrial}
\date{\today}

\begin{document}
\maketitle
\section{Filosofía del Sistema}
En ingeniería de datos, la interfaz gráfica (GUI) es un lujo, no una necesidad. Los servidores que procesan terabytes de datos agronómicos son \textit{headless}.
\section{Automatización con Bash}
El script \texttt{deploy.sh} utiliza un bucle \texttt{for} y \textit{Heredocs} para generar código dinámicamente. Esto permite escalar de 3 a 1000 sensores cambiando un solo número.
\end{document}
