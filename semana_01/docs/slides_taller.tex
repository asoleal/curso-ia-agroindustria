\documentclass{beamer}
\usetheme{Madrid}
\usecolortheme{beaver}
\usepackage[utf8]{inputenc}
\usepackage[spanish]{babel}
\usepackage{listings}
\usepackage{xcolor}

\title{Taller 01: Dominando la Terminal}
\subtitle{Guía Paso a Paso}
\author{Ingeniería de Software Agroindustrial}
\date{\today}

\begin{document}

\begin{frame}
    \titlepage
\end{frame}

\begin{frame}{Objetivo de la Sesión}
    \textbf{Misión:} Convertirnos en administradores de sistemas.
    \vspace{0.5cm}
    \begin{enumerate}
        \item Aprender a movernos sin mouse.
        \item Manipular archivos de datos reales.
        \item Automatizar la creación de infraestructura con un Script.
        \item Guardar nuestro trabajo profesionalmente con Git.
    \end{enumerate}
\end{frame}

% --- PARTE 1 ---
\section{Parte 1: Reconocimiento}

\begin{frame}[fragile]{1. ¿Dónde estoy?}
    Abre tu terminal. Lo primero es ubicarse.
    
    \begin{block}{Comando: pwd}
    \begin{verbatim}
$ pwd
/home/estudiante/curso_ia/semana_01
    \end{verbatim}
    \end{block}
    
    \textbf{Acción:} Escribe \texttt{pwd} y verifica que estás en la carpeta correcta.
\end{frame}

\begin{frame}[fragile]{2. ¿Qué hay aquí?}
    Vamos a mirar alrededor.
    
    \begin{block}{Comando: ls (Listar)}
    \begin{verbatim}
$ ls -F
datos/   TALLER_TERMINAL.md   test_entorno.py
    \end{verbatim}
    \end{block}
    
    \textbf{Acción:}
    \begin{itemize}
        \item Ejecuta \texttt{ls -F} (La 'F' te muestra qué es carpeta y qué es archivo).
        \item Ejecuta \texttt{ls datos/} para ver qué hay dentro de esa carpeta sin entrar.
    \end{itemize}
\end{frame}

% --- PARTE 2 ---
\section{Parte 2: Manipulación de Datos}

\begin{frame}[fragile]{3. Leyendo el CSV}
    Encontraste un archivo \texttt{produccion\_lote.csv}. Vamos a leerlo.
    
    \begin{block}{Comandos: cat y head}
    \begin{verbatim}
$ cat datos/produccion_lote.csv
(Muestra todo el contenido)

$ head -n 2 datos/produccion_lote.csv
(Muestra solo las primeras 2 lineas)
    \end{verbatim}
    \end{block}
\end{frame}

\begin{frame}[fragile]{4. Backup de Seguridad}
    Antes de tocar nada, hacemos una copia.
    
    \begin{block}{Copiar (cp) y Mover (mv)}
    \begin{verbatim}
# 1. Crear copia
$ cp datos/produccion_lote.csv datos/backup.csv

# 2. Crear carpeta de seguridad
$ mkdir respaldos

# 3. Mover la copia a la carpeta
$ mv datos/backup.csv respaldos/
    \end{verbatim}
    \end{block}
\end{frame}

% --- PARTE 3 ---
\section{Parte 3: El Reto de Automatización}

\begin{frame}{El Reto Final}
    \begin{alertblock}{Escenario}
    El jefe quiere configurar \textbf{3 zonas de monitoreo}.
    \begin{itemize}
        \item Crear carpeta \texttt{zona\_1}, \texttt{zona\_2}, \texttt{zona\_3}.
        \item Dentro de cada una, carpeta \texttt{sensores}.
        \item Dentro, un script \texttt{main.py} simulando datos.
    \end{itemize}
    \end{alertblock}
    \textbf{Prohibido hacerlo a mano.} Debemos crear un script \texttt{deploy.sh}.
\end{frame}

\begin{frame}[fragile]{5. El Script (deploy.sh)}
    Usa \texttt{nano deploy.sh} y escribe esto:
    
    \begin{lstlisting}[language=bash, basicstyle=\tiny]
#!/bin/bash
echo "--- INICIANDO DESPLIEGUE ---"

for i in {1..3}
do
    echo "Configurando Zona $i..."
    mkdir -p "zona_$i/sensores"
    
    # Generar Python automaticamente
    cat << FIN > "zona_$i/sensores/main.py"
import random
t = random.uniform(20, 35)
print(f"Zona $i: {t:.1f} C")
FIN

done
    \end{lstlisting}
\end{frame}

\begin{frame}[fragile]{6. Ejecución y Verificación}
    Para correr tu script:
    
    \begin{block}{Ejecutar}
    \begin{verbatim}
# 1. Dar permisos (Importante)
$ chmod +x deploy.sh

# 2. Correr
$ ./deploy.sh
    \end{verbatim}
    \end{block}
    
    Si todo sale bien, verás los mensajes de creación. ¡Felicidades, has automatizado tu primer proceso!
\end{frame}

\end{document}
