\documentclass[12pt, a4paper]{article}
\usepackage[utf8]{inputenc}
\usepackage[spanish]{babel}
\usepackage{geometry}
\usepackage{xcolor}
\usepackage{listings}
\usepackage{tcolorbox}
\usepackage{hyperref}
\usepackage{longtable}
\usepackage{booktabs}
\usepackage{amssymb} % Necesario para simbolos matematicos

\geometry{top=2.5cm, bottom=2.5cm, left=2.5cm, right=2.5cm}

% Configuración de colores estilo VS Code
\definecolor{codegreen}{rgb}{0,0.6,0}
\definecolor{codegray}{rgb}{0.5,0.5,0.5}
\definecolor{codepurple}{rgb}{0.58,0,0.82}
\definecolor{backcolour}{rgb}{0.95,0.95,0.92}

\lstdefinestyle{mystyle}{
    backgroundcolor=\color{backcolour},   
    commentstyle=\color{codegreen},
    keywordstyle=\color{magenta},
    numberstyle=\tiny\color{codegray},
    stringstyle=\color{codepurple},
    basicstyle=\ttfamily\footnotesize,
    breaklines=true,
    captionpos=b,                    
    keepspaces=true,                 
    numbers=left,                    
    numbersep=5pt,                  
    showspaces=false,                
    showstringspaces=false,
    showtabs=false,                  
    tabsize=2
}
\lstset{style=mystyle}

\title{\textbf{Manual de Ingeniería de Software I}\\ Fundamentos de Terminal, Linux y Automatización}
\author{Semana 01 - Curso de IA Agroindustrial}
\date{\today}

\begin{document}

\maketitle
\tableofcontents
\newpage

\section{Introducción: La Filosofía CLI}
La Interfaz de Línea de Comandos (CLI) no es una herramienta antigua; es una herramienta \textbf{precisa}. En entornos de servidores, computación en la nube (AWS, Azure) y dispositivos IoT (Raspberry Pi en campo), la interfaz gráfica consume recursos innecesarios.
\\
\\
\textbf{Principio Fundamental:} En Linux, \textit{todo es un archivo}. Un sensor es un archivo, el disco duro es un archivo, y los procesos son archivos. Manipular archivos es manipular el sistema.

\section{Módulo 1: Navegación y Reconocimiento}
Antes de dar órdenes, debemos saber dónde estamos.

\subsection{El Sistema de Archivos}
\begin{itemize}
    \item \textbf{/} (Raíz): El inicio de todo.
    \item \textbf{\textasciitilde} (Home): Tu carpeta personal (/home/usuario).
    \item \textbf{.} (Punto): El directorio actual.
    \item \textbf{..} (Doble punto): El directorio padre (atrás).
\end{itemize}

\subsection{Comandos de Exploración}
\begin{longtable}{p{0.3\textwidth} p{0.6\textwidth}}
\toprule
\textbf{Comando} & \textbf{Descripción Técnica} \\
\midrule
\texttt{pwd} & \textit{Print Working Directory}. Muestra la ruta absoluta actual. \\
\texttt{ls} & \textit{List}. Muestra archivos en la carpeta actual. \\
\texttt{ls -l} & Formato largo. Muestra permisos, dueño, tamaño y fecha. \\
\texttt{ls -a} & Muestra archivos ocultos (los que inician con punto). \\
\texttt{ls -R} & Recursivo. Muestra el contenido de todas las subcarpetas. \\
\texttt{ls -lh} & Human-readable. Muestra tamaños en KB, MB, GB. \\
\texttt{history} & Muestra la lista de comandos que has ejecutado recientemente. \\
\texttt{clear} & Limpia la pantalla (pero no borra el historial). \\
\bottomrule
\end{longtable}

\section{Módulo 2: Manipulación de Infraestructura}
Como ingenieros, creamos y destruimos estructuras de datos.

\subsection{Creación y Destrucción}
% CORRECCION: Se reemplazo el emoji por $\triangle$
\begin{tcolorbox}[colback=red!5!white,colframe=red!75!black,title=$\triangle$ Cuidado con rm]
El comando \texttt{rm} no envía a la papelera. \textbf{Elimina permanentemente}.
\end{tcolorbox}

\begin{itemize}
    \item \texttt{mkdir nombre}: Crea una carpeta.
    \item \texttt{mkdir -p a/b/c}: Crea una ruta completa de carpetas anidadas.
    \item \texttt{touch archivo.txt}: Crea un archivo vacío o actualiza la fecha de uno existente.
    \item \texttt{rm archivo}: Borra un archivo.
    \item \texttt{rm -r carpeta}: Borra una carpeta y su contenido (Recursivo).
    \item \texttt{rm -rf carpeta}: Borra forzadamente sin preguntar (Force).
\end{itemize}

\subsection{Movimiento y Copiado}
\begin{itemize}
    \item \texttt{cp origen destino}: Copia un archivo.
    \item \texttt{cp -r origen destino}: Copia una carpeta entera.
    \item \texttt{mv origen destino}: Mueve un archivo. También se usa para \textbf{renombrar}.
\end{itemize}

\section{Módulo 3: Visualización y Edición de Datos}
Un Data Scientist a menudo necesita inspeccionar CSVs de gigabytes sin abrir Excel.

\subsection{Lectura Eficiente}
\begin{itemize}
    \item \texttt{cat archivo}: Concatena e imprime todo el archivo en pantalla.
    \item \texttt{head -n 5 archivo}: Muestra solo las primeras 5 líneas.
    \item \texttt{tail -n 5 archivo}: Muestra las últimas 5 líneas (útil para logs en tiempo real).
    \item \texttt{less archivo}: Permite navegar el archivo con flechas (salir con 'q').
\end{itemize}

\subsection{El Editor Nano}
Nano es un editor de texto en terminal.
\begin{itemize}
    \item \textbf{Ctrl + O}: Guardar (Write Out).
    \item \textbf{Ctrl + X}: Salir.
    \item \textbf{Ctrl + K}: Cortar línea.
    \item \textbf{Ctrl + U}: Pegar línea.
\end{itemize}

\section{Módulo 4: Scripting y Automatización (Bash)}
Bash es un lenguaje de programación interpretado. Nos permite automatizar tareas repetitivas.

\subsection{Estructura de un Script}
Todo script debe iniciar con el \textit{Shebang}: \texttt{\#!/bin/bash}. Esto le dice al sistema qué intérprete usar.

\subsection{Variables y Bucles}
\begin{lstlisting}[language=bash]
#!/bin/bash
# Definicion de variables
SENSOR="Humedad"

# Bucle FOR para iterar rangos
for i in {1..5}; do
    echo "Leyendo $SENSOR en Sector $i..."
    # Aqui irian comandos de lectura real
done
\end{lstlisting}

\subsection{Permisos de Ejecución}
Linux protege el sistema impidiendo que cualquier archivo se ejecute como programa.
\begin{itemize}
    \item \texttt{chmod +x script.sh}: Otorga permiso de e\textbf{x}ecution (ejecución).
\end{itemize}

\section{Módulo 5: Git (Control de Versiones)}
Git gestiona la historia de tu código.

\begin{enumerate}
    \item \textbf{git init}: Crea un repositorio en la carpeta actual.
    \item \textbf{git status}: ¿Qué ha cambiado desde la última "foto"?
    \item \textbf{git add .}: Prepara todos los cambios para la foto.
    \item \textbf{git commit -m "mensaje"}: Toma la foto y la guarda.
    \item \textbf{git log}: Muestra el historial de fotos.
    \item \textbf{git push}: Sube las fotos a la nube (GitHub/GitLab).
\end{enumerate}

\end{document}
