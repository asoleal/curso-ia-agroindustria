\documentclass[12pt, a4paper]{article}
\usepackage[utf8]{inputenc}
\usepackage[T1]{fontenc}
\usepackage[spanish]{babel}
\usepackage{geometry}
\usepackage{xcolor}
\usepackage{listings}
\usepackage{tcolorbox}
\usepackage[colorlinks=true, linkcolor=blue, urlcolor=blue, citecolor=black]{hyperref}
\usepackage{longtable}
\usepackage{booktabs}
\usepackage{amssymb}
\usepackage{enumitem}
\usepackage{graphicx}
\usepackage{fancyhdr}

\geometry{top=2.5cm, bottom=2.5cm, left=2.5cm, right=2.5cm}

% --- Configuración de Colores y Estilo de Código ---
\definecolor{codegreen}{rgb}{0,0.6,0}
\definecolor{codegray}{rgb}{0.5,0.5,0.5}
\definecolor{codepurple}{rgb}{0.58,0,0.82}
\definecolor{backcolour}{rgb}{0.97,0.97,0.97}

\lstdefinestyle{mystyle}{
    backgroundcolor=\color{backcolour},   
    commentstyle=\color{codegreen},
    keywordstyle=\color{magenta},
    numberstyle=\tiny\color{codegray},
    stringstyle=\color{codepurple},
    basicstyle=\ttfamily\footnotesize,
    breaklines=true,
    captionpos=b,
    keepspaces=true,
    numbers=left,
    numbersep=5pt,
    showspaces=false,
    showstringspaces=false,
    showtabs=false,
    tabsize=2,
    inputencoding=utf8,
    extendedchars=true,
    literate={á}{{\'a}}1 {é}{{\'e}}1 {í}{{\'i}}1 {ó}{{\'o}}1 {ú}{{\'u}}1 {ñ}{{\~n}}1
}
\lstset{style=mystyle}

% --- Configuración de Encabezado y Pie de Página ---
\pagestyle{fancy}
\fancyhf{}
\rhead{Curso de IA Agroindustrial}
\lhead{Semana 01: Terminal, Linux y Git}
\rfoot{\thepage}

\title{\textbf{Manual de Ingeniería de Software I}\\ Fundamentos de Terminal, Linux, Automatización y Git}
\author{John Jairo Leal Gómez \\ Curso de IA Aplicada al Agro}
\date{\today}

\begin{document}

\maketitle
\tableofcontents
\newpage

\section{Introducción: La Filosofía CLI}
La Interfaz de Línea de Comandos (CLI) no es una herramienta antigua; es una herramienta \textbf{precisa}. En la era de la IA, la terminal es esencial porque vive en:
\begin{itemize}
    \item \textbf{Servidores en la nube} (AWS, Google Cloud) donde se entrenan modelos.
    \item \textbf{Dispositivos IoT} en campo (Raspberry Pi, sensores).
    \item \textbf{Entornos de desarrollo} como GitHub Codespaces o Docker.
\end{itemize}

\subsection*{Principio Fundamental: Todo es un archivo}
En Linux, \textit{todo es un archivo}. Un sensor es un archivo, el disco duro es un archivo, y los procesos son archivos. Manipular archivos es manipular el sistema.

\section{Módulo 1: Navegación y Reconocimiento}
Antes de dar órdenes, debemos saber dónde estamos.

\subsection{El Sistema de Archivos}
\begin{itemize}
    \item \textbf{/} (Raíz): El inicio de todo.
    \item \textbf{\textasciitilde} (Home): Tu carpeta personal (/home/usuario).
    \item \textbf{.} (Punto): El directorio actual.
    \item \textbf{..} (Doble punto): El directorio padre (atrás).
\end{itemize}

\subsection{Comandos de Exploración}
\begin{longtable}{p{0.3\textwidth} p{0.6\textwidth}}
\toprule
\textbf{Comando} & \textbf{Descripción Técnica} \\
\midrule
\texttt{pwd} & Muestra la ruta absoluta actual. \\
\texttt{ls} & Lista los archivos en la carpeta actual. \\
\texttt{ls -l} & Formato largo (permisos, dueño, tamaño y fecha). \\
\texttt{ls -a} & Muestra archivos ocultos (configuraciones como .git). \\
\texttt{ls -R} & Recursivo. Muestra el contenido de todas las subcarpetas. \\
\texttt{ls -lh} & \textit{Human-readable}. Muestra tamaños en KB, MB, GB. \\
\texttt{history} & Muestra la lista de comandos recientes. \\
\texttt{clear} & Limpia la pantalla (pero no borra el historial). \\
\bottomrule
\end{longtable}

\begin{tcolorbox}[colback=blue!5!white,colframe=blue!75!black,title=Consejo Profesional]
Usa la tecla \texttt{Tab} para autocompletar nombres de archivos y carpetas. ¡Ahorra tiempo y evita errores!
\end{tcolorbox}

\section{Módulo 2: Manipulación de Infraestructura}
Como ingenieros, creamos y destruimos estructuras de datos y directorios.

\subsection{Creación y Destrucción}
\begin{itemize}
    \item \texttt{mkdir nombre}: Crea una carpeta.
    \item \texttt{mkdir -p a/b/c}: Crea una ruta completa de carpetas anidadas.
    \item \texttt{touch archivo.txt}: Crea un archivo vacío o actualiza la fecha.
    \item \texttt{rm archivo}: Borra un archivo.
    \item \texttt{rm -r carpeta}: Borra una carpeta y su contenido (Recursivo).
    \item \texttt{rm -rf carpeta}: Borra forzadamente sin preguntar (Force).
\end{itemize}

\begin{tcolorbox}[colback=red!5!white,colframe=red!75!black,title=$\triangle$ Cuidado con rm]
El comando \texttt{rm} no envía a la papelera. \textbf{Elimina permanentemente}. Nunca uses \texttt{rm -rf /}.
\end{tcolorbox}

\subsection{Movimiento y Copiado}
\begin{itemize}
    \item \texttt{cp origen destino}: Copia un archivo.
    \item \texttt{cp -r origen destino}: Copia una carpeta entera.
    \item \texttt{mv origen destino}: Mueve un archivo. También se usa para \textbf{renombrar}.
\end{itemize}

\section{Módulo 3: Visualización y Edición de Datos}
Un Ingeniero de Datos a menudo necesita inspeccionar CSVs de gigabytes sin abrir Excel.

\subsection{Lectura Eficiente}
\begin{itemize}
    \item \texttt{cat archivo}: Concatena e imprime todo el archivo en pantalla.
    \item \texttt{head -n 5 archivo}: Muestra solo las primeras 5 líneas.
    \item \texttt{tail -n 5 archivo}: Muestra las últimas 5 líneas (útil para logs en tiempo real).
    \item \texttt{less archivo}: Permite navegar el archivo con flechas (salir con 'q').
\end{itemize}

\subsection{El Editor Nano}
Nano es un editor de texto en terminal, fundamental para ediciones rápidas en servidores.
\begin{itemize}
    \item \textbf{Ctrl + O}: Guardar (Write Out).
    \item \textbf{Ctrl + X}: Salir.
    \item \textbf{Ctrl + K}: Cortar línea.
    \item \textbf{Ctrl + U}: Pegar línea.
\end{itemize}

\section{Módulo 4: Scripting y Automatización (Bash)}
Bash es un lenguaje interpretado que nos permite automatizar tareas repetitivas.

\subsection{Estructura y Ejemplo Real}
Todo script debe iniciar con el \textit{Shebang}: \texttt{\#!/bin/bash}. El siguiente ejemplo automatiza la creación de zonas de monitoreo:

\begin{lstlisting}[language=bash,caption={deploy.sh: Automatización de Infraestructura}]
#!/bin/bash
# Script para desplegar zonas de monitoreo

echo ">>> Iniciando despliegue automatico..."

# Bucle FOR para crear 3 zonas
for i in {1..3}; do
    echo "Configurando Zona $i"
    mkdir -p "zona_$i/sensores"
    
    # Generar script de Python dinamicamente (Heredoc)
    cat << EOF > "zona_$i/sensores/main.py"
print(f"Zona $i: Reporte simulado")
EOF
done

echo "[OK] Despliegue completado."
\end{lstlisting}

\subsection{Permisos de Ejecución}
Linux protege el sistema impidiendo que cualquier archivo se ejecute como programa. Para correr tu script debes otorgar permisos:
\begin{itemize}
    \item \texttt{chmod +x deploy.sh}: Otorga permiso de e\textbf{x}ecution (ejecución).
\end{itemize}

\section{Módulo 5: Git (Control de Versiones)}
Git gestiona la historia de tu código, permitiendo "viajar en el tiempo" si algo sale mal.

\begin{enumerate}
    \item \textbf{git init}: Crea un repositorio en la carpeta actual.
    \item \textbf{git status}: ¿Qué ha cambiado desde la última "foto"?.
    \item \textbf{git add .}: Prepara todos los cambios para la foto.
    \item \textbf{git commit -m "mensaje"}: Toma la foto y la guarda permanentemente.
    \item \textbf{git log}: Muestra el historial de commits.
    \item \textbf{git push}: Sube tu código a la nube (GitHub/GitLab).
\end{enumerate}

\end{document}
