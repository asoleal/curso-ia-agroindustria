\documentclass[10pt, aspectratio=169]{beamer}

% =====================================================
% PAQUETES Y CONFIGURACIÓN
% =====================================================
\usepackage{fontspec}
\usepackage[spanish, es-tabla, es-noquoting]{babel}
\usepackage{amsmath, amsfonts, amssymb}
\usepackage{listings}
\usepackage{xcolor}
\usepackage{graphicx}
\usepackage{tikz}
\usepackage{hyperref}
\usepackage{tcolorbox}

% =====================================================
% TEMA Y COLORES
% =====================================================
\usetheme{Madrid}
\usecolortheme{default}

% Colores personalizados
\definecolor{primary}{RGB}{0, 51, 102}
\definecolor{accent}{RGB}{46, 139, 87}
\definecolor{danger}{RGB}{180, 0, 0}
\definecolor{terminalbg}{RGB}{28, 28, 28}
\definecolor{terminalfg}{RGB}{135, 255, 0}

% Colores del tema
\setbeamercolor{palette primary}{bg=primary, fg=white}
\setbeamercolor{palette secondary}{bg=accent, fg=white}
\setbeamercolor{palette tertiary}{bg=white, fg=primary}
\setbeamercolor{alerted text}{fg=danger}
\setbeamercolor{block title}{bg=accent, fg=white}
\setbeamercolor{block body}{bg=accent!10, fg=black}

% =====================================================
% ESTILO DE CÓDIGO
% =====================================================
\lstdefinestyle{bashstyle}{
  backgroundcolor=\color{terminalbg},
  basicstyle=\ttfamily\tiny\color{terminalfg},
  breaklines=true,
  language=bash,
  showstringspaces=false,
  keywordstyle=\color{cyan},
  commentstyle=\color{gray!70}\itshape,
  belowskip=0pt,
  aboveskip=0pt
}

% =====================================================
% METADATOS
% =====================================================
\title{\textbf{Ingeniería de IA I}}
\subtitle{Fundamentos de Ciencia de Datos:\\
Terminal, Git, Python y Docker}
\author{AgroFuture AI Training}
\date{Enero 2026}
\institute{Departamento de Ingeniería \\ Valle del Cauca, Colombia}

\begin{document}

% =====================================================
% PORTADA
% =====================================================
\begin{frame}[plain]
  \titlepage
\end{frame}

% =====================================================
% TABLA DE CONTENIDOS
% =====================================================
\begin{frame}{Índice}
  \tableofcontents
\end{frame}

% =====================================================
% SECCIÓN 1: INTRODUCCIÓN
% =====================================================
\section{Introducción: ¿Por Qué Estas Herramientas?}

\begin{frame}{Realidad del Científico de Datos}
  \centering
  \begin{tikzpicture}[scale=1.5]
    \draw[fill=accent!20] (0,0) rectangle (2,3);
    \node at (1, 2.5) {\small 3 horas};
    \node at (1, 2.0) {Limpieza};

    \draw[fill=accent!40] (2.2,0) rectangle (4.2,2);
    \node at (3.2, 1.5) {\small 2 horas};
    \node at (3.2, 1.0) {Automatización};

    \draw[fill=accent!60] (4.4,0) rectangle (6.4,1);
    \node at (5.4, 0.5) {\small 1 hora};
    \node at (5.4, 0.0) {Modelos};
  \end{tikzpicture}

  \vspace{0.5cm}
  {\large \textbf{Por cada hora entrenando modelos:}}
  \begin{itemize}
    \item 3 horas de limpieza y validación de datos
    \item 2 horas de automatización y scripting
    \item 1 hora de documentación y versionado
  \end{itemize}
\end{frame}

\begin{frame}{Principios Fundamentales}
  \begin{block}{Reproducibilidad}
    Un modelo que no puede reproducirse no es ciencia: es magia.
  \end{block}

  \vspace{1cm}

  \begin{alertblock}{¿Qué aprenderemos?}
    \begin{enumerate}
      \item \textbf{Terminal Linux}: navegar, procesar datos
      \item \textbf{Git}: versionado y control de cambios
      \item \textbf{Python}: automatización y análisis
      \item \textbf{Docker}: ambientes reproducibles
    \end{enumerate}
  \end{alertblock}
\end{frame}

% =====================================================
% SECCIÓN 2: TERMINAL LINUX
% =====================================================
\section{Fase 1: Terminal y Bash}

\begin{frame}{Navegación Básica}
  \begin{itemize}
    \item \texttt{pwd}: saber en qué carpeta estás actualmente
    \item \texttt{ls -lh}: listar archivos con tamaños legibles
    \item \texttt{cd carpeta}: cambiar a una carpeta
    \item \texttt{mkdir -p ruta}: crear carpetas (y subcarpetas)
  \end{itemize}

  \vspace{0.5cm}

  \begin{block}{Objetivo}
    Aprender a orientarse en el sistema de archivos de un proyecto de datos.
  \end{block}
\end{frame}

\begin{frame}[fragile]{Navegación Básica: ejemplo}
  \begin{exampleblock}{Sesión típica}
\begin{lstlisting}[style=bashstyle]
cd ~/proyecto_ia
mkdir -p data/raw data/processed
ls -lh data/
\end{lstlisting}
  \end{exampleblock}

  \vspace{0.3cm}

  \small
  Esto crea carpetas para datos brutos y procesados, y verifica que existan.
\end{frame}

\begin{frame}{Manipulación de Archivos}
  \begin{itemize}
    \item \texttt{cp archivo.txt copia.txt}: copiar (usa \texttt{-i} para preguntar)
    \item \texttt{mv viejo.txt nuevo.txt}: mover o renombrar
    \item \texttt{rm -i archivo.txt}: eliminar con confirmación
    \item \texttt{du -sh *}: ver tamaño de carpetas
    \item \texttt{wc -l datos.csv}: contar filas
  \end{itemize}
\end{frame}

\begin{frame}[fragile]{Manipulación de Archivos: ejemplo}
  \begin{exampleblock}{Organizar proyecto}
\begin{lstlisting}[style=bashstyle]
cd ~/proyecto_ia
mkdir -p scripts reports
cp ~/sensores.csv data/raw/sensores.csv
mv README_tmp.md README.md
\end{lstlisting}
  \end{exampleblock}

  \vspace{0.3cm}

  \small
  Las operaciones básicas permiten organizar un proyecto profesional.
\end{frame}

\begin{frame}{Pipelines con Pipes}
  El operador \texttt{|} encadena comandos: salida de uno = entrada del siguiente.

  \vspace{0.5cm}

  \begin{itemize}
    \item \texttt{cut}: extrae columnas
    \item \texttt{sort}: ordena filas
    \item \texttt{uniq}: elimina duplicados
  \end{itemize}

  \vspace{0.5cm}

  \textbf{Ventaja:} sin crear archivos intermedios, sin abrir editores gráficos.
\end{frame}

\begin{frame}[fragile]{Pipelines con Pipes: ejemplo}
  \begin{exampleblock}{Análisis en una línea}
\begin{lstlisting}[style=bashstyle]
# Contar registros únicos por sensor
cut -d',' -f1 sensores.csv | sort | uniq -c

# Ver las 5 temperaturas más altas
cut -d',' -f3 sensores.csv | sort -nr | head -5
\end{lstlisting}
  \end{exampleblock}
\end{frame}

\begin{frame}{Procesamiento con awk}
  \texttt{awk} es un mini-lenguaje para cálculos y manipulación en datos tabulares.

  \vspace{0.5cm}

  \begin{itemize}
    \item Extrae columnas específicas
    \item Filtra filas por condiciones
    \item Calcula sumas, promedios, máximos
    \item Genera reportes directamente
  \end{itemize}
\end{frame}

\begin{frame}[fragile]{Procesamiento con awk: ejemplos}
  \begin{exampleblock}{Cálculos estadísticos}
\begin{lstlisting}[style=bashstyle]
# Temperatura promedio
awk -F',' '{s+=$3} END {print s/NR}' sensores.csv

# Contar registros con humedad < 30
awk -F',' '$4 < 30 {count++} END {print count}' sensores.csv

# Guardar filas con humedad > 80
awk -F',' '$4 > 80 {print}' sensores.csv > alertas.csv
\end{lstlisting}
  \end{exampleblock}
\end{frame}

% =====================================================
% SECCIÓN 3: GIT
% =====================================================
\section{Fase 2: Git y Versionado}

\begin{frame}{¿Por Qué Git?}
  \begin{columns}
    \column{0.5\textwidth}
    \textbf{Sin Git:}
    \begin{itemize}
      \item \texttt{modelo\_v1.pth}
      \item \texttt{modelo\_v2\_final.pth}
      \item \texttt{modelo\_v2\_final2.pth}
      \item ¿Quién cambió qué?
      \item ¿Cuándo?
    \end{itemize}

    \column{0.5\textwidth}
    \textbf{Con Git:}
    \begin{itemize}
      \item Una versión actual
      \item Historial completo
      \item Quién hizo qué
      \item Cuándo
      \item Por qué (mensaje)
    \end{itemize}
  \end{columns}
\end{frame}

\begin{frame}{Flujo Básico de Git}
  \begin{enumerate}
    \item \texttt{git init}: inicializar repositorio
    \item \texttt{git status}: ver qué cambió
    \item \texttt{git add .}: seleccionar cambios
    \item \texttt{git commit -m "mensaje"}: guardar con descripción
    \item \texttt{git log --oneline}: ver historial resumido
  \end{enumerate}
\end{frame}

\begin{frame}[fragile]{Flujo Básico de Git: ejemplo}
  \begin{exampleblock}{Primera sesión completa}
\begin{lstlisting}[style=bashstyle]
cd proyecto_ia
git init
git add .
git commit -m "Inicial: estructura del proyecto"
git log --oneline
\end{lstlisting}
  \end{exampleblock}

  \vspace{0.3cm}

  \small
  Así comienza cualquier proyecto profesional de ciencia de datos.
\end{frame}

\begin{frame}{Buenas Prácticas: .gitignore}
  \textbf{¿Qué NO debe versionarse?}
  \begin{itemize}
    \item \texttt{data/raw/}: datos brutos sin procesar
    \item \texttt{*.pth, *.h5}: modelos entrenados (muy grandes)
    \item \texttt{venv/}: entornos virtuales
    \item \texttt{.ipynb\_checkpoints/}: archivos temporales
    \item \texttt{\_\_pycache\_\_/}: cache de Python
  \end{itemize}
\end{frame}

\begin{frame}[fragile]{Buenas Prácticas: .gitignore (crear)}
  \begin{exampleblock}{Crear archivo .gitignore}
\begin{lstlisting}[style=bashstyle]
cat > .gitignore << EOF
venv/
data/raw/
*.pth
.ipynb_checkpoints/
__pycache__/
EOF

git add .gitignore
git commit -m "Configura .gitignore"
\end{lstlisting}
  \end{exampleblock}
\end{frame}

\begin{frame}{Mensajes de Commit Claros}
  \begin{alertblock}{Evitar}
    \begin{itemize}
      \item \texttt{"cambios"}
      \item \texttt{"arreglado"}
      \item \texttt{"v2"}
    \end{itemize}
  \end{alertblock}

  \begin{exampleblock}{Hacer}
    \begin{itemize}
      \item \texttt{"Añade validación de valores faltantes"}
      \item \texttt{"Corrige filtro de humedad en CSV"}
      \item \texttt{"Implementa pipeline de limpieza automática"}
    \end{itemize}
  \end{exampleblock}
\end{frame}

% =====================================================
% SECCIÓN 4: PYTHON Y ENTORNOS
% =====================================================
\section{Fase 3: Python y Entornos Virtuales}

\begin{frame}{Entornos Virtuales: Problema}
  \begin{itemize}
    \item Proyecto A necesita \texttt{pandas 1.0}
    \item Proyecto B necesita \texttt{pandas 2.0}
    \item \texttt{pip install pandas==2.0} rompe Proyecto A
  \end{itemize}

  \vspace{1cm}

  \begin{block}{Solución}
    Cada proyecto su propio entorno aislado.
  \end{block}
\end{frame}

\begin{frame}[fragile]{Entornos Virtuales: crear}
  \begin{exampleblock}{Crear un entorno}
\begin{lstlisting}[style=bashstyle]
python -m venv ia_env
source ia_env/bin/activate
pip install pandas numpy scikit-learn
pip freeze > requirements.txt
\end{lstlisting}
  \end{exampleblock}

  \vspace{0.3cm}

  \small
  \texttt{requirements.txt} guarda las versiones exactas instaladas.
\end{frame}

\begin{frame}[fragile]{Entornos Virtuales: reproducir}
  \begin{exampleblock}{Recrear el mismo entorno en otra máquina}
\begin{lstlisting}[style=bashstyle]
python -m venv ia_env
source ia_env/bin/activate
pip install -r requirements.txt
\end{lstlisting}
  \end{exampleblock}

  \vspace{0.3cm}

  \small
  Garantiza que otros colaboradores y servidores tengan las mismas versiones.
\end{frame}

\begin{frame}[fragile]{Pipeline Shell + Python}
  \begin{exampleblock}{Combinar lo mejor de ambos mundos}
\begin{lstlisting}[style=bashstyle]
#!/bin/bash
source ia_env/bin/activate

# Procesar con shell
cut -d',' -f1,3,4 data/raw/sensores.csv > temp.csv

# Analizar con Python
python - << EOF
import pandas as pd
df = pd.read_csv('temp.csv')
print(df.describe())
EOF
\end{lstlisting}
  \end{exampleblock}
\end{frame}

% =====================================================
% SECCIÓN 5: DOCKER
% =====================================================
\section{Fase 4: Docker}

\begin{frame}{¿Por Qué Docker?}
  \begin{columns}
    \column{0.5\textwidth}
    \textbf{Mi máquina:}
    \begin{itemize}
      \item Python 3.10
      \item pandas 2.0
      \item Funciona
    \end{itemize}

    \column{0.5\textwidth}
    \textbf{En el servidor:}
    \begin{itemize}
      \item Python 3.8
      \item pandas 1.5
      \item ¡Error!
    \end{itemize}
  \end{columns}

  \vspace{1cm}

  \centering
  \textbf{Docker: "Empaqueta tu ambiente completo"}
\end{frame}

\begin{frame}{Conceptos Básicos de Docker}
  \begin{block}{Imagen}
    Plantilla inmutable: \texttt{FROM python:3.10} + librerías + código
  \end{block}

  \begin{block}{Contenedor}
    Instancia en ejecución de una imagen
  \end{block}

  \begin{block}{Dockerfile}
    Archivo que describe cómo construir la imagen
  \end{block}

  \begin{block}{Registry}
    Lugar donde publicas imágenes (p.ej. Docker Hub)
  \end{block}
\end{frame}

\begin{frame}[fragile]{Dockerfile para Ciencia de Datos}
  \begin{exampleblock}{Ejemplo mínimo}
\begin{lstlisting}[style=bashstyle]
FROM python:3.10-slim
WORKDIR /app
COPY requirements.txt .
RUN pip install -r requirements.txt
COPY . .
EXPOSE 8888
CMD ["jupyter", "lab", "--ip=0.0.0.0"]
\end{lstlisting}
  \end{exampleblock}
\end{frame}

\begin{frame}{Partes de un Dockerfile}
  \begin{itemize}
    \item \texttt{FROM}: imagen base oficial (p.ej. \texttt{python:3.10-slim})
    \item \texttt{WORKDIR}: directorio de trabajo dentro del contenedor
    \item \texttt{COPY}: copiar archivos locales
    \item \texttt{RUN}: instalar dependencias (pip install, apt-get, etc.)
    \item \texttt{EXPOSE}: puertos que expone (documentación)
    \item \texttt{CMD}: comando por defecto al iniciar
  \end{itemize}
\end{frame}

\begin{frame}[fragile]{Construcción y Ejecución}
  \begin{exampleblock}{Construir la imagen}
\begin{lstlisting}[style=bashstyle]
docker build -t proyecto_ia:latest .
\end{lstlisting}
  \end{exampleblock}

  \vspace{0.3cm}

  \begin{exampleblock}{Ejecutar un contenedor}
\begin{lstlisting}[style=bashstyle]
docker run --rm \
  -v $(pwd)/data:/app/data \
  -v $(pwd)/reports:/app/reports \
  proyecto_ia:latest
\end{lstlisting}
  \end{exampleblock}
\end{frame}

\begin{frame}{Opciones de docker run}
  \begin{itemize}
    \item \texttt{-v carpeta\_local:/app/carpeta}: montar carpeta local dentro del contenedor
    \item \texttt{--rm}: borrar contenedor al terminar (limpieza)
    \item \texttt{-p 8888:8888}: exponer puerto (para Jupyter)
    \item \texttt{-it}: modo interactivo (para shells)
  \end{itemize}

  \vspace{0.5cm}

  \small
  Los datos en tu máquina siempre están seguros; el contenedor solo ve lo que le permitas.
\end{frame}

\begin{frame}[fragile]{Docker Compose: Múltiples Servicios}
  \begin{exampleblock}{docker-compose.yml}
\begin{lstlisting}[style=bashstyle]
version: "3.8"
services:
  db:
    image: postgres:13
  notebook:
    build: .
    depends_on:
      - db
    ports:
      - "8888:8888"
\end{lstlisting}
  \end{exampleblock}

  \vspace{0.3cm}

  \begin{lstlisting}[style=bashstyle]
docker compose up    # Levanta ambos
docker compose down  # Los apaga
\end{lstlisting}
\end{frame}

% =====================================================
% SECCIÓN 6: ESTRUCTURA DE PROYECTO
% =====================================================
\section{Estructura de Proyecto Profesional}

\begin{frame}{Carpetas Recomendadas}
  \begin{columns}
    \column{0.5\textwidth}
    \begin{itemize}
      \item \texttt{data/raw/}: datos sin procesar
      \item \texttt{data/processed/}: datos limpios
      \item \texttt{scripts/}: código reutilizable
      \item \texttt{notebooks/}: exploración Jupyter
    \end{itemize}

    \column{0.5\textwidth}
    \begin{itemize}
      \item \texttt{models/}: modelos entrenados
      \item \texttt{reports/}: figuras y tablas
      \item \texttt{README.md}: descripción
      \item \texttt{requirements.txt}: dependencias
    \end{itemize}
  \end{columns}

  \vspace{0.5cm}

  \begin{alertblock}{Regla fundamental}
    Si no está versionado en Git, no existe.
  \end{alertblock}
\end{frame}

\begin{frame}[fragile]{Crear Estructura Completa}
  \begin{exampleblock}{Una sola orden}
\begin{lstlisting}[style=bashstyle]
mkdir -p proyecto_ia/{data/raw,data/processed,\
scripts,notebooks,models,reports}

cd proyecto_ia
git init
touch README.md
git add .
git commit -m "Inicial: estructura"
\end{lstlisting}
  \end{exampleblock}
\end{frame}

% =====================================================
% SECCIÓN 7: FLUJO COMPLETO
% =====================================================
\section{Flujo Completo de un Proyecto}

\begin{frame}{De Cero a Análisis}
  \begin{enumerate}
    \item Crear estructura con \texttt{mkdir -p}
    \item \texttt{git init} + crear \texttt{.gitignore}
    \item \texttt{python -m venv ia\_env}
    \item Descargar datos a \texttt{data/raw/}
    \item Scripts de limpieza en \texttt{scripts/}
    \item Análisis en notebooks
    \item Reportes en \texttt{reports/}
    \item Commits frecuentes con mensajes claros
    \item Crear \texttt{Dockerfile} para reproducibilidad
  \end{enumerate}
\end{frame}

\begin{frame}{Checklist Profesional}
  \begin{itemize}
    \item[\checkmark] Código versionado en Git
    \item[\checkmark] Datos en carpeta \texttt{raw/}, nunca modificar
    \item[\checkmark] Scripts reproducibles, no manuales
    \item[\checkmark] Entorno virtual con \texttt{requirements.txt}
    \item[\checkmark] README explicando el proyecto
    \item[\checkmark] Dockerfile para ejecutar en cualquier máquina
    \item[\checkmark] Datos sensibles en \texttt{.gitignore}
    \item[\checkmark] Commits pequeños y frecuentes
  \end{itemize}
\end{frame}

% =====================================================
% SECCIÓN 8: RETO FINAL
% =====================================================
\section{Reto Final}

\begin{frame}{Proyecto Completo: Monitoreo de Sensores}
  Construye un proyecto que:

  \begin{enumerate}
    \item Descargue datos de sensores (archivo CSV)
    \item Valide datos: cuente filas, busque faltantes
    \item Limpie datos automáticamente
    \item Genere gráficos y tablas
    \item Guarde resultados en \texttt{reports/}
    \item Todo versionado en Git
    \item Todo reproducible en Docker
  \end{enumerate}

  \vspace{0.5cm}

  \begin{alertblock}{Requisito: Idempotencia}
    Ejecutar 10 veces debe dar el mismo resultado sin errores.
  \end{alertblock}
\end{frame}

\begin{frame}{Componentes del Reto}
  \begin{block}{Script de descarga}
    \texttt{scripts/descargar\_datos.sh}: obtiene CSV de fuente confiable
  \end{block}

  \begin{block}{Script de validación}
    \texttt{scripts/validar\_datos.py}: estadísticas y calidad
  \end{block}

  \begin{block}{Script de limpieza}
    \texttt{scripts/limpiar\_datos.py}: elimina errores, guarda en \texttt{processed/}
  \end{block}

  \begin{block}{Dockerfile}
    Ejecuta todo automáticamente en un contenedor
  \end{block}
\end{frame}

% =====================================================
% CIERRE
% =====================================================
\section{Siguientes Pasos}

\begin{frame}{Nivel Intermedio}
  \textbf{Herramientas complementarias:}
  \begin{itemize}
    \item MLflow: experiment tracking y reproducibilidad
    \item GitHub Actions: CI/CD automático
    \item DVC: versionado de datos grandes
    \item Airflow: orquestación de pipelines
  \end{itemize}
\end{frame}

\begin{frame}{Nivel Avanzado}
  \textbf{Productividad y escalabilidad:}
  \begin{itemize}
    \item FastAPI: servir modelos como APIs
    \item Kubernetes: orquestación de contenedores en producción
    \item Prometheus + Grafana: monitoreo
    \item GitLab CI/CD o Jenkins: pipelines complejos
  \end{itemize}
\end{frame}

\begin{frame}{Recordar}
  \centering

  \Large
  \textbf{La reproducibilidad es lo más importante.}

  \vspace{1cm}

  \normalsize
  Un código sin documentación es deuda técnica. \\
  Un modelo sin versión es imposible de auditar. \\
  Datos sin respaldo desaparecen.

  \vspace{1cm}

  \small
  \textbf{Las herramientas que aprendiste}
  garantizan que tu trabajo sea sólido y duradero.
\end{frame}

\begin{frame}[plain]
  \centering
  \LARGE \textbf{¿Preguntas?}

  \vspace{1.5cm}

  \normalsize
  Documentación oficial:
  \begin{itemize}
    \item Bash: \url{https://www.gnu.org/software/bash/manual/}
    \item Git: \url{https://git-scm.com/doc}
    \item Docker: \url{https://docs.docker.com/}
    \item Python venv: \url{https://docs.python.org/3/library/venv.html}
  \end{itemize}

  \vspace{1cm}

  \Large
  \textbf{¡Muchas gracias!}
\end{frame}

\end{document}
