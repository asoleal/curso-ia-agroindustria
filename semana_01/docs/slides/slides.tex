\documentclass{beamer}

% --- Paquetes y Configuración ---
\usepackage{fontspec}
\usepackage[spanish]{babel}
\setmainfont{DejaVu Sans} % o cualquier fuente sans-serif legible
\usepackage{listings}
\usepackage{xcolor}

% --- Configuración del Tema y Colores ---
\usetheme{Madrid}
\usecolortheme{beaver} % Base rojiza/gris, la ajustaremos
\definecolor{DarkGreen}{rgb}{0.0, 0.4, 0.0}
\definecolor{TechGrey}{rgb}{0.3, 0.3, 0.3}

% Personalización de colores para "Agro-IA"
\setbeamercolor{structure}{fg=DarkGreen}
\setbeamercolor{frametitle}{bg=TechGrey, fg=white}
\setbeamercolor{title}{bg=DarkGreen, fg=white}

% Configuración de bloques de código para Beamer
\definecolor{codegreen}{rgb}{0,0.6,0}
\definecolor{backcolour}{rgb}{0.95,0.95,0.92}

\lstdefinestyle{mystyle}{
    backgroundcolor=\color{backcolour},
    commentstyle=\color{codegreen},
    keywordstyle=\color{blue},
    basicstyle=\ttfamily\scriptsize,
    breaklines=true,
    captionpos=b,
    keepspaces=true,
    showstringspaces=false,
    tabsize=2,
    literate={á}{{\'a}}1 {é}{{\'e}}1 {í}{{\'i}}1 {ó}{{\'o}}1 {ú}{{\'u}}1 {ñ}{{\~n}}1
}
\lstset{style=mystyle}

% --- Información de la Presentación ---
\title[Ingeniería de Software I]{Fundamentos de Terminal, Linux y Automatización}
\subtitle{Semana 01: Del GUI a la CLI en la Era de la IA}
\author{Curso de IA Agroindustrial}
\institute{Facultad de Ingeniería}
\date{\today}

\titlegraphic{\includegraphics[width=2.5cm]{logo_unal.png}}
\logo{\includegraphics[height=0.8cm]{logo_unal.png}}
\begin{document}

% 1. Portada
\begin{frame}
    \titlepage
\end{frame}

% 2. Índice
\begin{frame}{Agenda del Día}
    \tableofcontents
\end{frame}

% --- Sección 1: Introducción ---
\section{Introducción: La Filosofía CLI}

\begin{frame}{¿Por qué la Terminal en 2024?}
    \begin{block}{La Realidad de la IA}
    La Inteligencia Artificial no vive en Windows ni en macOS con interfaz gráfica. Vive en:
    \end{block}

    \begin{itemize}
        \item \textbf{Servidores Cloud:} AWS, Google Cloud, Azure (Linux sin monitor).
        \item \textbf{IoT y Edge Computing:} Raspberry Pi, sensores en campo, drones.
        \item \textbf{Contenedores:} Docker y Kubernetes.
    \end{itemize}

    \vspace{0.5cm}
    \centering
    \textbf{\textit{``Si puedes hacerlo en la terminal, puedes automatizarlo.''}}
\end{frame}

\begin{frame}{Principio Fundamental}
    \begin{alertblock}{Todo es un archivo}
    En Linux/Unix, todo se representa como un archivo o un directorio.
    \end{alertblock}
    \vspace{0.3cm}
    \begin{itemize}
        \item ¿El disco duro? Es un archivo (\texttt{/dev/sda}).
        \item ¿La temperatura del CPU? Es un archivo.
        \item ¿La memoria RAM? Es un archivo.
    \end{itemize}
\end{frame}

% --- Sección 2: Navegación ---
\section{Módulo 1: Navegación}

\begin{frame}[fragile]{Dónde estamos y cómo nos movemos}
    Comandos esenciales para no perderse:

    \begin{description}
        \item[\texttt{pwd}] \textbf{P}rint \textbf{W}orking \textbf{D}irectory. ¿Dónde estoy?
        \item[\texttt{ls -lh}] Listar archivos con detalles y tamaño legible.
        \item[\texttt{cd ..}] Ir al directorio anterior (padre).
        \item[\texttt{cd \textasciitilde}] Ir a mi carpeta personal (Home).
    \end{description}

    \begin{exampleblock}{Tip de Oro: Autocompletado}
    ¡Usa la tecla \textbf{TAB}!
    Escribe \texttt{cd Doc} + \textbf{TAB} $\to$ \texttt{cd Documentos/}
    \end{exampleblock}
\end{frame}

% --- Sección 3: Manipulación ---
\section{Módulo 2: Gestión de Archivos}

\begin{frame}[fragile]{Crear y Organizar}
    La estructura de un proyecto de IA empieza aquí:

    \begin{lstlisting}[language=bash]
# Crear carpeta del proyecto
mkdir proyecto_ia

# Crear estructura anidada (flag -p)
mkdir -p proyecto_ia/datos/raw

# Crear archivo vacio
touch proyecto_ia/readme.md
    \end{lstlisting}
\end{frame}

\begin{frame}{La Zona de Peligro: Borrado}
    \begin{alertblock}{\textbf{ADVERTENCIA CRÍTICA}}
    En la terminal \textbf{NO EXISTE LA PAPELERA DE RECICLAJE}.
    \end{alertblock}

    \begin{itemize}
        \item \texttt{rm archivo.txt} $\to$ Se fue para siempre.
        \item \texttt{rm -rf carpeta/} $\to$ Borra carpeta y contenido sin preguntar.
    \end{itemize}

    \vspace{0.5cm}
    \textbf{Regla:} Piensa dos veces antes de presionar Enter con \texttt{rm}.
\end{frame}

% --- Sección 4: Datos y Edición ---
\section{Módulo 3: Visualización de Datos}

\begin{frame}[fragile]{Inspección de Datos (Big Data)}
    No puedes abrir un CSV de 5GB en Excel. Usa esto:

    \begin{itemize}
        \item \textbf{\texttt{head -n 5 data.csv}}: Ver las primeras 5 filas (cabeceras).
        \item \textbf{\texttt{tail -f server.log}}: Ver el final del archivo en \textit{tiempo real}.
        \item \textbf{\texttt{cat archivo}}: Ver todo el contenido (cuidado si es grande).
        \item \textbf{\texttt{less archivo}}: Navegar con flechas (salir con `q`).
    \end{itemize}
\end{frame}

% --- Sección 5: Scripting ---
\section{Módulo 4: Automatización (Bash)}

\begin{frame}[fragile]{Automatización con Bash}
    Ejemplo: Script \texttt{deploy.sh} para crear 3 zonas de monitoreo.

    \begin{lstlisting}[language=bash]
#!/bin/bash
echo "Iniciando despliegue..."

for i in {1..3}; do
    echo "Creando Zona $i"
    mkdir -p "zona_$i/sensores"

    # Crear script Python automaticamente
    cat << EOF > "zona_$i/sensores/main.py"
print(f"Zona $i: Lista")
EOF
done
    \end{lstlisting}

    \small{Recuerda dar permisos: \texttt{chmod +x deploy.sh}}
\end{frame}

% --- Sección 6: Git ---
\section{Módulo 5: Git}

\begin{frame}{Control de Versiones con Git}
    El ``Botón de Guardar'' de los ingenieros.

    \begin{enumerate}
        \item \textbf{\texttt{git init}}: ``Aquí empiezo a trackear cambios''.
        \item \textbf{\texttt{git add .}}: ``Apartame estos cambios para la foto''.
        \item \textbf{\texttt{git commit -m ``msj''}}: ``Toma la foto y guárdala''.
        \item \textbf{\texttt{git push}}: ``Sube la foto a la nube''.
        \item \textbf{\texttt{git status}}: ``¿Cómo está mi proyecto?''.
    \end{enumerate}
\end{frame}

% --- Sección 7: Extras Importantes ---
\section{Extras y Trucos}

\begin{frame}{Atajos de Teclado (Salvavidas)}
    Memoriza esto para triplicar tu velocidad:

    \begin{columns}
        \column{0.5\textwidth}
        \begin{block}{Control}
        \begin{itemize}
            \item \textbf{Ctrl + C}: Abortar/Matar comando actual.
            \item \textbf{Ctrl + L}: Limpiar pantalla (clear).
            \item \textbf{Ctrl + D}: Cerrar terminal (exit).
        \end{itemize}
        \end{block}

        \column{0.5\textwidth}
        \begin{block}{Búsqueda}
        \begin{itemize}
            \item \textbf{Flecha Arriba}: Ver comando anterior.
            \item \textbf{Ctrl + R}: Buscar en el historial (Reverse search). ¡Muy útil!
        \end{itemize}
        \end{block}
    \end{columns}
\end{frame}

\begin{frame}{Resumen Final}
    \centering
    \Large
    \textbf{Semana 1 Completada} \\
    \vspace{0.5cm}
    \normalsize
    Ya tienen las bases: Moverse, Crear, Automatizar y Versionar. \\
    \vspace{0.5cm}
    \textit{¿Preguntas?}
\end{frame}

\end{document}
