\documentclass[11pt, letterpaper]{article}

% --- PAQUETES DE CONFIGURACIÓN ---
\usepackage[utf8]{inputenc}
\usepackage[spanish]{babel}
\usepackage[T1]{fontenc}
\usepackage{geometry}
\geometry{top=2.5cm, bottom=2.5cm, left=2.5cm, right=2.5cm}
\usepackage{xcolor}
\usepackage{hyperref}
\usepackage{enumitem}
\usepackage{titlesec}
\usepackage{fancyhdr}

% --- COLORES Y ESTILO ---
\definecolor{primaryColor}{RGB}{0, 102, 51} % Verde Agro
\definecolor{secondaryColor}{RGB}{40, 40, 40}

\hypersetup{
    colorlinks=true,
    linkcolor=primaryColor,
    urlcolor=primaryColor,
}

% --- DATOS DEL DOCUMENTO ---
\title{\textbf{\Large Plan de Curso} \\ \Huge \textbf{Fundamentos de IA Aplicados a Procesos Agroindustriales}}
\author{John Jairo Leal Gómez \\ \textit{Enfoque: Toma de decisiones basada en datos}}
\date{\today}

% --- INICIO DEL DOCUMENTO ---
\begin{document}

\maketitle
\thispagestyle{empty}
\hrule

\section{Descripción del Curso}
Este curso introduce a los estudiantes de agroindustria en los conceptos fundamentales de la Inteligencia Artificial (IA) y su aplicación práctica en la transformación y procesamiento de materias primas. 

A través de un enfoque práctico, los estudiantes aprenderán a identificar oportunidades donde la IA puede optimizar la calidad, el rendimiento y la eficiencia de los procesos agroalimentarios. Se utilizarán herramientas computacionales modernas como soporte técnico para materializar estos conceptos.

\section{Objetivos de Aprendizaje}
Al finalizar el curso, el estudiante estará en capacidad de:
\begin{itemize}[label=\textcolor{primaryColor}{\textbullet}]
    \item \textbf{Comprender los fundamentos de la IA:} Distinguir entre IA, Machine Learning (Aprendizaje Automático) y Deep Learning, desmitificando su uso en el sector agroindustrial.
    \item \textbf{Identificar variables críticas:} Reconocer qué datos de un proceso industrial (temperatura, pH, brix, tiempo) son valiosos para entrenar modelos inteligentes.
    \item \textbf{Aplicar modelos predictivos:} Utilizar algoritmos básicos para predecir comportamientos en procesos de alimentos (ej. predicción de vida útil o curvas de secado).
    \item \textbf{Operar herramientas de soporte:} Manejar competentemente un entorno de trabajo digital (GitHub, Python, LaTeX) necesario para desplegar y documentar soluciones de IA.
\end{itemize}

\newpage

\section{Contenido Programático (8 Semanas)}

\subsection*{Módulo 1: Herramientas Computacionales (El ``Taller Digital'') - Semanas 1-2}
\textit{Objetivo: Nivelación técnica necesaria para trabajar con IA.}

\begin{description}
    \item[Semana 1: Entorno de Trabajo y Gestión de Versiones.] \hfill \\
    Configuración de GitHub Codespaces. ¿Por qué la IA necesita muchos datos y versiones ordenadas? Estructura de un proyecto de datos agroindustrial.
    \item[Semana 2: Python como Lenguaje de la IA.] \hfill \\
    Fundamentos rápidos de Python enfocados en datos: Listas, Bibliotecas (Pandas) y manejo básico de Datasets (archivos CSV de procesos). Introducción rápida a LaTeX para reportes.
\end{description}

\subsection*{Módulo 2: Fundamentos de IA y Datos (La Teoría Aplicada) - Semanas 3-5}
\textit{Objetivo: Entender cómo ``aprenden'' las máquinas a partir de procesos agroindustriales.}

\begin{description}
    \item[Semana 3: Del Dato a la Información.] \hfill \\
    ¿Qué es el ``Entrenamiento''? Limpieza de datos de sensores industriales. Diferencia entre datos estructurados (tablas de laboratorio) y no estructurados (imágenes de frutos).
    \item[Semana 4: Aprendizaje Supervisado (Regresión).] \hfill \\
    Concepto de predicción numérica. 
    \textit{Caso Práctico:} Predicción de grados Brix finales en una mermelada basándose en tiempos de cocción y temperatura inicial.
    \item[Semana 5: Aprendizaje Supervisado (Clasificación).] \hfill \\
    Concepto de categorización automática.
    \textit{Caso Práctico:} Algoritmo para clasificar lotes de materia prima como ``Aceptado'' o ``Rechazado'' según variables físico-químicas de entrada.
\end{description}

\subsection*{Módulo 3: Aplicación en Procesos Reales (Proyecto) - Semanas 6-8}
\textit{Objetivo: Resolver un problema agroindustrial usando lo aprendido.}

\begin{description}
    \item[Semana 6: Optimización de Procesos con IA.] \hfill \\
    Cómo la IA ayuda a reducir desperdicios y energía. Análisis de correlaciones: ¿Qué variable afecta más mi producto final?
    \item[Semana 7: Documentación de Resultados (LaTeX).] \hfill \\
    Interpretación de los modelos. Generación de un informe técnico científico que explique no solo el ``qué'' (resultado), sino el ``por qué'' (el modelo de IA).
    \item[Semana 8: Presentación de Proyectos.] \hfill \\
    Defensa de una solución de IA aplicada a un proceso específico (ej: Lácteos, Cárnicos, Frutas). El estudiante entrega: Código (Modelo) + Informe (LaTeX).
\end{description}

\section{Metodología de Evaluación}
Se evaluará la capacidad de aplicar el concepto de IA a un problema real, más que la memorización de código.

\begin{table}[h]
    \centering
    \begin{tabular}{|l|l|c|}
    \hline
    \textbf{Criterio} & \textbf{Descripción} & \textbf{Peso} \\ \hline
    Manejo de Herramientas & Uso correcto de Git y Python (Módulo 1) & 20\% \\ \hline
    Comprensión de IA & Explicación de modelos y selección de variables & 40\% \\ \hline
    Aplicación Práctica & Proyecto final (Solución a problema agroindustrial) & 40\% \\ \hline
    \end{tabular}
\end{table}

\end{document}