\documentclass[12pt, a4paper]{article}
\usepackage[utf8]{inputenc}
\usepackage[spanish]{babel}
\usepackage{amsmath}
\usepackage{geometry}

\geometry{top=2.5cm, bottom=2.5cm, left=2.5cm, right=2.5cm}

\title{\textbf{Manual de Ingeniería II}\\ Computación Vectorial y NumPy}
\author{Semana 02 - IA Agroindustrial}
\date{\today}

\begin{document}
\maketitle
\section{Arquitectura de Computadores}
NumPy es rápido debido a \textbf{SIMD} (Single Instruction, Multiple Data) y la localidad espacial de la memoria caché.
\section{Normalización Matemática}
Para entrenar redes neuronales, normalizamos los datos $X$ al rango $[0,1]$:
$$ X_{norm} = \frac{X - X_{min}}{X_{max} - X_{min}} $$
Esta operación se realiza mediante \textit{Broadcasting} en Python.
\end{document}
