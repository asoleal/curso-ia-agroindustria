\documentclass[aspectratio=169, 12pt]{beamer}

% --- PAQUETES Y FUENTES (XeLaTeX) ---
\usepackage{fontspec}
% Usamos fuentes seguras de Linux
\setsansfont{DejaVu Sans}[Scale=0.9] 
\setmonofont{DejaVu Sans Mono}[Scale=0.8]

\usepackage{polyglossia}
\setmainlanguage{spanish}

\usepackage{xcolor}
\usepackage{listings}
\usepackage{tikz}
\usetikzlibrary{shapes, arrows, positioning, babel} % 'babel' es vital para evitar conflictos

% --- TEMA Y COLORES ---
\usetheme{Madrid}
\usecolortheme{default}

% Colores personalizados (Igual que el manual)
\definecolor{primary}{RGB}{0, 85, 164}      % Azul Ingeniería
\definecolor{accent}{RGB}{34, 139, 34}      % Verde Agro
\definecolor{codebg}{RGB}{240, 240, 240}

% Aplicar colores al tema
\setbeamercolor{structure}{fg=primary}
\setbeamercolor{palette primary}{bg=primary, fg=white}
\setbeamercolor{palette secondary}{bg=primary!80!black, fg=white}
\setbeamercolor{palette tertiary}{bg=primary!60!black, fg=white}
\setbeamercolor{frametitle}{bg=primary, fg=white}
\setbeamercolor{block title}{bg=primary, fg=white}
\setbeamercolor{block body}{bg=blue!5!white, fg=black}
\setbeamercolor{alerted text}{fg=accent}

% --- CONFIGURACIÓN DE CÓDIGO (LITERATE PARA EMOJIS) ---
\lstdefinestyle{pythonstyle}{
    language=Python,
    backgroundcolor=\color{codebg},
    commentstyle=\color{gray}\itshape,
    keywordstyle=\color{primary}\bfseries,
    numberstyle=\tiny\color{gray},
    stringstyle=\color{accent},
    basicstyle=\ttfamily\footnotesize,
    breaklines=true,
    showstringspaces=false,
    frame=lines,
    % Truco para reemplazar emojis por texto seguro
    literate=
    {á}{{\'a}}1 {é}{{\'e}}1 {í}{{\'i}}1 {ó}{{\'o}}1 {ú}{{\'u}}1 {ñ}{{\~n}}1
    {⚠️}{{\textcolor{orange}{\textbf{!}}}}1
    {💧}{{\textcolor{blue}{\textbf{O}}}}1
    {✅}{{\textcolor{green}{\textbf{OK}}}}1
}
\lstset{style=pythonstyle}

% --- DATOS DE LA PRESENTACIÓN ---
\title[Ingeniería de Software I]{Fundamentos de Computación Científica}
\subtitle{Lógica Algorítmica y Vectorización para IA}
\author{Semana 02}
\institute{IA Aplicada al Agro}
\date{\today}

\begin{document}

% --- SLIDE 1: PORTADA ---
\begin{frame}
    \titlepage
    \centering
    \small \textit{"El código no solo debe funcionar, debe escalar."}
\end{frame}

% --- SLIDE 2: AGENDA ---
\begin{frame}{Hoja de Ruta}
    \tableofcontents
\end{frame}

% --- SECCIÓN 1 ---
\section{Lógica Defensiva}

\begin{frame}[fragile]{1. El Principio "Fail Fast"}
    En ingeniería, un error silencioso es peor que un error ruidoso.
    
    \begin{alertblock}{Problema: Arrow Code}
        Anidar \texttt{if} dentro de \texttt{if} hace el código ilegible y difícil de mantener.
    \end{alertblock}

\begin{lstlisting}
# MAL: Dificil de leer
def sistema_riego(sensor):
    if sensor.activo:
        if sensor.humedad < 30:
            activar_bombas()
\end{lstlisting}
\end{frame}

\begin{frame}[fragile]{Guard Clauses (Cláusulas de Guardia)}
    Validamos lo negativo primero y retornamos. El código "feliz" queda plano.

\begin{lstlisting}
# BIEN: Ingenieria de Software
def sistema_riego(sensor):
    # 1. Validacion (Fail Fast)
    if not sensor.activo:
        return "Error: Sensor inactivo"
        
    # 2. Logica de Negocio
    if sensor.humedad < 30:
        activar_bombas()
        return "Regando..."
        
    return "En espera"
\end{lstlisting}
\end{frame}

% --- SECCIÓN 2 ---
\section{Memoria y Complejidad}

\begin{frame}{2. Listas vs. Arrays (La verdad de la RAM)}
    ¿Por qué Python puro es lento? Porque las listas son punteros dispersos.
    
    \vspace{0.5cm}
    \centering
    \begin{tikzpicture}[scale=0.8, transform shape]
        % Lista Python
        \node at (0, 2) {\textbf{Python List}};
        \foreach \x in {0, 1.2, 2.4} {
            \draw[fill=gray!20] (\x,0) rectangle (\x+0.8,0.5);
            \draw[->, thick] (\x+0.4, 0.25) -- (\x+0.4, -0.8);
            \draw[fill=blue!10] (\x+rand*0.3, -1.5) circle (0.3);
        }
        
        % Flecha separadora
        \draw[->, ultra thick, primary] (3.5, 0) -- (4.5, 0);

        % NumPy Array
        \node at (6.5, 2) {\textbf{NumPy Array}};
        \draw[fill=primary!20] (5, -0.5) rectangle (8, 0.5);
        \node at (5.5,0) {1};
        \node at (6.5,0) {2};
        \node at (7.5,0) {3};
        \node[align=center] at (6.5, -1.5) {\textbf{Bloque Contiguo}\\(SIMD Ready)};
    \end{tikzpicture}
\end{frame}

\begin{frame}{Complejidad Computacional (Big O)}
    Analizar 1 millón de plantas.
    
    \begin{columns}
        \column{0.5\textwidth}
        \textbf{Python Loop (Iterativo)}
        \begin{itemize}
            \item Verifica tipo de dato en cada paso.
            \item No usa caché del CPU.
            \item Tiempo: $\approx 300ms$
        \end{itemize}
        
        \column{0.5\textwidth}
        \textbf{NumPy (Vectorizado)}
        \begin{itemize}
            \item Operación en C optimizado.
            \item Usa instrucciones SIMD.
            \item Tiempo: $\approx 5ms$
        \end{itemize}
    \end{columns}
    
    \vspace{0.5cm}
    \centering
    \textcolor{accent}{\textbf{¡NumPy es 60x más rápido!}}
\end{frame}

% --- SECCIÓN 3 ---
\section{Vectorización Práctica}

\begin{frame}[fragile]{3. Pensar en Matrices, no en Bucles}
    \textbf{Objetivo:} Encontrar zonas de sequía en un campo de $100 \times 100$.

\begin{lstlisting}
import numpy as np

# Generar datos simulados
campo = np.random.rand(100, 100)

# MALA PRACTICA (Estilo C antiguo)
# for i in range(100):
#     for j in range(100): ...

# BUENA PRACTICA (Vectorizacion)
# Crea una mascara booleana instantanea
zonas_sequia = campo < 0.3 

print(f"Hectareas afectadas: {np.sum(zonas_sequia)}")
\end{lstlisting}
\end{frame}

\begin{frame}[fragile]{Lógica Condicional: np.where}
    El equivalente a \texttt{=SI()} de Excel, pero masivo.
    
    $$ R = \text{np.where}(\text{condición}, \text{Valor SI}, \text{Valor NO}) $$

\begin{lstlisting}
# Calculo de riego necesario
# Si humedad < 0.4, echar agua hasta llegar a 0.5
# Si no, echar 0 agua.

agua = np.where(campo < 0.4, 0.5 - campo, 0)
\end{lstlisting}
\end{frame}

% --- CIERRE ---
\begin{frame}{Próximos Pasos}
    \begin{block}{Semana 03: Big Data}
    \begin{itemize}
        \item \textbf{Pandas:} Manejo de archivos CSV gigantes.
        \item \textbf{Matplotlib:} Visualización de datos.
        \item \textbf{Limpieza:} Manejo de datos corruptos (NaN).
    \end{itemize}
    \end{block}
    
    \vspace{1cm}
    \centering
    \Large \textbf{¿Preguntas?}
\end{frame}

\end{document}
